%=====================================================
%  Template File for Journal of Symbolic Computation
%=====================================================

\documentclass[final,1p,times,authoryear]{elsarticle}
\usepackage{amsmath}
\usepackage{amsthm}
\usepackage{amssymb}
\usepackage{amsfonts}

\newtheorem{theorem}{Theorem}
\newtheorem{lemma}[theorem]{Lemma}
\newtheorem{corollary}[theorem]{Corollary}
\newtheorem{conjecture}[theorem]{Conjecture}
\newtheorem{proposition}[theorem]{Proposition}
\newtheorem{definition}[theorem]{Definition}
\newtheorem{notation}[theorem]{Notation}
\newtheorem{example}[theorem]{Example}
\newtheorem{remark}[theorem]{Remark}
\newtheorem{problem}[theorem]{Problem}
\newtheorem{acknowledgment}[]{Acknowledgment}
\iffalse
\newenvironment{proof}{\noindent{\em Proof:}}{$\Box$~\\}
\fi


\usepackage{comment}
\usepackage{url}
\def\d{{\fontencoding{T1}\selectfont\dj}}
\def\D{{\fontencoding{T1}\selectfont\DJ}}

\begin{document}

\begin{frontmatter}

\title{Title goes here.....\footnote{This work was partially supported by the Serbian Ministry of Science grant 174021}}

\author{Filip Mari\' c}
\address{School of Mathematics, University of Belgrade, Studentski trg 16, 11000, Belgrade, Serbia}
\ead{filip@matf.bg.ac.rs}
\ead[url]{http://www.matf.bg.ac.rs/~filip/}

\author{Danijela Simi\' c}
\address{School of Mathematics, University of Belgrade, Studentski trg 16, 11000, Belgrade, Serbia}
\ead{danijela@matf.bg.ac.rs}
\ead[url]{http://www.matf.bg.ac.rs/~danijela/}

\begin{abstract}
Abstract goes here.....
\end{abstract}

\begin{keyword}
Keywords go here.... 
\end{keyword}
\end{frontmatter}


\section{Introduction}
For over two millennia, geometry has been one of the major topics in
education all around the world. Since Euclid's ,,Elements'', geometry
has been a central field for introducing students to deduction and
rigorous argumentation. 

In recent years, computers and technology have been intensively used
to change how geometry is taught. Dynamic geometry systems such as
GeoGebra\footnote{\url{https://www.geogebra.org/}},
Cinderella\footnote{\url{https://www.cinderella.de/tiki-index.php}},
Geometer's Sketchpad\footnote{\url{http://www.dynamicgeometry.com/}},
Cabri\footnote{\url{http://www.cabri.com/}},
Eukleides\footnote{\url{http://www.eukleides.org/}} are now regularly used
in all levels of education. Students use such systems to perform
geometric constructions, and obtain diagrams that can be distored by
moving free points. Such dynamic diagrams are better than static
images, since moving free points can shed additional light to the
problem, reveal degenerate cases, help student to determine if
something is true only if some special order of points is considered
(for example, some property could be true only if a point is between
some other two points, and be false if that is not the case, some
property could be true only for accute, and not for obtuse angles
etc.).

By doing extensive distortion of diagrams by moving free points,
student can be pretty sure if a property is generaly true (i.e., true
in all, but a small number of degenerate cases), but still, that
cannot be considered to be a proof, and is error prone. Therefore,
recently dynamic geometry systems have been extended by automated
reasoning systems, that can automatically prove statements about
constructed objects \cite{geogebra-provers}. Such theorem provers are
usually algebraic (they perform calculations on symbolic coordinates
of geometric objects).

% reci zasto se u 3d prostoru svojstva ne vide golim oko toliko dobro koliko u 2d
In our view, application of dynamic geometry software for three
dimension space is even more important since it is often hard to
determine some properties by observing diagram. This happens due to
fact that three dimensional space is presented as two dimensional
projection and thus measures cannot be given as in original diagram.

% Ovde reci nesto o DGS za 3d - Geogebra, Cinderella, Stereos
Some geometry systems started developing support for three dimension
constructions. The new version of GeoGeobra have developed support for
dynamic three dimension
graphics\footnote{\url{https://wiki.geogebra.org/en/3D_Graphics_View}}. It
is possible to create and dynamically change three dimension objects
such as points, lines, polygons and spheres, as well as three
dimension function graphics. However, this system does not support
proving statements about three dimension objects. Both Cinderella and
Cindy3D\footnote{\url{http://gagern.github.io/Cindy3D/}} have
extensions for drawing three dimension objects using commands and
formulas describing them.

Most recent research in both dynamic geometry systems and automated
theorem proving in geometry has been devoted only to two--dimensional
Euclidean geometry (plane geometry). Although there have been some
limited attempts to apply algebraic theorem proving methods to
three--dimensional Euclidean space geometry (solid geometry), we are
not aware that there are thorough descriptions of these methods, nor
publicly available implementations of automated provers for solid
geometry. 

Having this in mind, this work tries to bridge several gaps that we
feel are present in current state-of-the-art in the field of automatic
proving in solid geometry.
\begin{enumerate}
\item In this paper we examine and compare several ways of applying
  algebraic theorem provers based on Gr\"obner bases and Wu's method
  to solid geometry problems.

\item We describe integration of algebraic theorem provers with
  dynamic geometry systems for solid geometry.

\item We provide an implementation of one dynamic geometry system for
  solid geometry, capable of proving statements about the properties
  of constructed objects.

\item We also analyze a corpus of problems from solid geometry and
  evaluate methods to those problems.

\item We discuss challenges and possible applications in the field of
  geometry education.
\end{enumerate}

% \paragraph{Overview of the paper.}

\section{Related work}
\paragraph{Automated reasoning in plane geometry.}
Automated theorem proving has a history more than fifty years
long. Most successful automated theorem methods are those for proving
in geometry. Automated theorem provers for geometry are based either
on the synthetic approach or on the algebraic approach (based on some
coordinates).

Major breakthrough in the algebraic approach was made by
\cite{wu}. The geometric construction and the statement are first
encoded as a set of polynomial equalities over their coordinates, and
then the Wu's method uses algebraic techniques for dealing with these
polynomials. In his paper \cite{ritt} described similar ideas as
Wu, so nowdays this method is usually called Ritt-Wu's method. Many
theorems were successfully proved using this method
\cite{chou1984}. The success of the Wu's method motivated other
researcher to develop new methods, and one of the most prominent is
the Gr\"obner basis method based on Buchberger's algorithm
\cite{buchberger}. Similar to Wu's method, the Gr\"obner basis also
reasons representation in form of a set of polynomial
equalities. There are many implementations of these algorithms and
some are in commercial software (e.g. Matllab, Mathematica or
Maple). However, both methods produce only yes or no answers instead
of human understandable proofs and cannot deal with width inequality
(therefore cannot reason about the order of points).

Several coordinate-free methods that do not use coordinate
representation of points are developed in the 1980's. The reason for
their development was in the idea that it is possible to develop
prover that would produce human readable proofs. Most prominent are
the area method \cite{area} and the full angle method
\cite{fullangle}, and those methods are usually referred to as
semi--algebraic, since they involve reasoning over some special
geometric quantities (e.g., signed area or Pythagora's
differences). These methods are usually not as efficient as algebraic
methods and have smaller scope than algebraic methods.

One of the first synthetic provers, that uses method of resolution,
was developed by \cite{gelertner} whose idea was to develop
the proof that is similar to human proof. He successfully proved many
problems taken from high--school textbooks. In modern times, synthetic
provers are usually based on coherent (synthetic) logic --- a special
fragment of first order logic convenient for geometric reasoning. For
example, the prover ArgoCLP \cite{sana} uses coherent logic to produce
human readable proofs.


\paragraph{Automated reasoning in solid geometry.}
\cite{volumemethod} at al.~presented volume method.  It is a
semi--algebraic method that is extension of the area method for solid
geometry. Hypotheses can be described constructively and conclusions
are polynomial equations of several geometry quantities, such as
volumes, ratios of line segments, ratios of areas, and Pythagoras
differences. The key idea of the method is to eliminate points from
the conclusion of a geometry statement using several basic
propositions about volumes.

\cite{shao2016challenging} used solid geometry problems from
Mathematical Olympiad problems and exercises to evaluate different
geometry provers.  In the paper are presented three different
examples. For each example are given polynomials (derived using
pen--and--paper) representing solid geometry conjecture. Using these
examples is demonstrated that algebraic methods can be used for
proving in solid geometry. For each example they used three different
methods, characteristic set method \cite{wu}, Gr\"obner basis method
\cite{buchberger} and vector algebra method \cite{lord1985method}.
These methods are compered and they give conclusion that vector
algebra method gives better geometry proof but derived equations can
be long and difficult for manipulation and calculation. They tested
these methods on 97 problems, but they derived all polynomials using
pen--and--paper. Although Maple is used for Gr\"obner basis method,
they do not present any automatic algebraization of solid geometric
statements, e.g. how to transform geometric statements into
polynomials.

\cite{loch2009geometric} analyzed how algebraic methods could be
applied for reducing constraints for 3D buildings. Focus was on
analyzing existing city models and constraints such as parallelity,
orthogonality and symmetry.  The goal was to determine witch
constraints are subsumed by others and therefore can be omitted.  They
presented polynomial equations for parallelity, orthogonality and
incidence for point and plane and Wu's characteristic set method was
successfully applied. Since polynomials were quite complex the key was
to reduce the number of variables and to use miltilinear polynomials
rather than quadratic. They also made a prototype of the system where
user can manipulate with constraints.

To the best of our knowledge, there are no papers that describe
application of Wu's or Gr\"obner basis method into problems in solid
geometry.

\section{Background}
\subsection{Algebrization of geometry relations}
We shall assume that geometry problems are given in terms of relations
between the geometric object involved (points, lines, planes, circles,
\ldots). Each theorem under consideration is given in form of an
implication, and both hypothesis and conclusions are given as one or
more geometric relations. In some cases, the properties of constructed
geometric objects should be proved. The construction is usually
specified by a series of construction steps, and each construction
step is a function that builds new geometric objects out of the
existing ones. However, the properties of the new object can usually
be specified in terms of one or more relations, so without losing
generality, it can be assumed that we are dealing only with relations.

The standard algebrization procedure introduces fresh symbolic
variables for point coordinates and introduces (polynomial) equations
that characterize every relation between objected observed.


Each object can be given independently by adding its arbitrary
parameters or it can be given using relationship with other, already
introduced objects. In the second case, for each new object both the
parameters and the polynomials that represent relationship between new
object and other objects in the construction are
introduced. Polynomials contain parameters of all objects in the
construction rule and analytic identities. All polynomials are
recorded in construction set which is later used in GeoProver.

Beside construction rules, there are statement rules, rules used to
express some condition that can be check using GeoProver. For each of
these conditions are created polynomials which are recorded in the
statement set. Although in most of the cases, there is only one
statement polynomial, there were couple statement that generated more
than one. As mentioned earlier, GeoProver can only have one statement
rule. However, this not pose a problem, since for each polynomial in
the statement set GeoProver was applied (always using the same
construction set). The statement is considered to be true only if
GeoProver returned true for all polynomials in the statement set.

For the purpose of the further text we are going to assume that we
have all the necessary data for each construction or statement rule
(our system collects this data, but we are not going to explain
technical side of the system since we fill this is not the scope of
this paper). This means that we have all the parameters of already
given objects.

\subsection{Algebraic methods}

Algebraic methods in geometry are well described in literature
\cite{wu, buchberger}. In this section we give a brief account on
algebraic methods in geometry, in order to explain some relevant steps
needed in algebraization of solid objects.


Once the geometric theorem has been algebrized, it is possible to
apply algebraic theorem proving methods. During algebraization are
gained polynomials equations describing objects of the form $p(v_1,
\ldots, v_n) = q(v_1, \ldots, v_n)$. Before applying any
algebraization method, these equations are transformed to $p(v_1,
\ldots, v_n) - q(v_1, \ldots, v_n) = 0$. Lets denote polynomials (left
side of equations) representing relation between objects width $f_i$,
$i = 1, \ldots, k$, and lets denote relations to be proved with $g_j$,
$j = 1, \ldots, l$, then proving theorem reduces to checking if for
each $g_j$ it holds that
$$\forall v_1, \ldots, v_n \in \mathbb{R}. \bigwedge_{i = 1}^{k} f_i(v_1, \ldots, v_n) = 0 \Longrightarrow g_j(v_1, \ldots, v_n) = 0.$$

As Tarski suggested, proving geometric properties can be done using
quantifier elimination procedure for the reals, but this approach is
inefficient and proving nontrivial geometric properties would take too
long. But, there is another approach. The main insight, given by Wu
Wen-ts\"un in 1978, is that remarkably many geometrical theorems, when
formulated as universal algebraic statements in terms of coordinates,
are also true for all complex values of the “coordinates”. However, a
care should be taken, since, in some cases, the condition holds for
$\mathbb{R}$, but not for $\mathbb{C}$, and in these cases methods
fail due to counterexamples in $\mathbb{C}$. The statement to be
proved is following:
$$\forall v_1, \ldots, v_n \in \mathbb{C}. \bigwedge_{i = 1}^{k} f_i(v_1, \ldots, v_n) = 0 \Longrightarrow g_j(v_1, \ldots, v_n) = 0.$$

This is true when $g_j$ belongs to the radical ideal $I = \langle f_1,
\ldots, f_k \rangle$, i.e. when exists an integer $r$ and polynomials
$h_1, \ldots, h_k$ such that $g_j^r = \sum_{i=1}^k h_if_i$.

The two most significant algebraic methods that use a kind of
Euclidean division to check the validity of an observed conjecture are
Buchberger's method consists in transforming the generating set into a
Gr\"obner basis and the Wu's method.

\paragraph{Wu's method}

The first step of simple Wu's method \cite{wu} uses the
pseudo--division operation to transform the polynomial system to
triangular form, i.e. to a system of equations where each successive
equation introduces exactly one dependent variable. After that, the
final reminder is calculated by pseudo dividing polynomial for the
statements to be proved by each polynomial from triangular system.
Summarizing, Wu's method, in its simplest form, allows to compute some
polynomials $c, h_1, \ldots, h_k$ and $r$ such that
$$cg_j = \sum_{i=1}^{k}h_if_i + r$$ If the final remainder $r$ is
equal to zero, then the conjecture is considered to be proved. This
simple method of Wu is not complete (in algebraic sense) and
sometimes the result of applying method is indecisive, i.e. the
theorem can't be proved nor disproved. A more complex and complete
version of the method uses ascending chains which are considered in
the Ritt-Wu principle.

\paragraph{Gr\"obner basis method}

$G$ is gr\"obner basis for ideal $I = \langle f_1, \ldots, f_k
\rangle$ if and only if multivariate division any polynomial belonging
to ideal $I$ with $G$ gives $0$. This means that proving given
conjecture consists of two steps. The first is determining the
Gr\"obner basis for polynomials describing relation between objects,
$f_1, \ldots, f_k$. The second step is multivariate division of
polynomial describing relation to be proved, $g_j$ with $G$. If the
result of division is $0$ then statement is proved, otherwise is
disproved.

For calculating Gr\"obner basis is used Buchberger's algorithm witch
transforms a given set of polynomials into a Gr\"obner basis with
respect to some monomial order using B-reduction.

\subsection{Implementations of algebraic methods}

For testing results of applied agrebaization we used three systems --
GeoProver that uses simple Wu's method \cite{geoprover},
Mathematica\footnote{\url{https://www.wolfram.com/mathematica/}} with
implemented Gr\"obner basis method and
Maple\footnote{\url{https://www.maplesoft.com/products/maple/}} with
implemented Wu's method \cite{wsolve}. Shortly, all systems are going
to be presented here.

\paragraph{GeoProver}

GeoProver is open source software implemented in Java providing
support for algebraic theorem proving using Wu's method. GeoProver
consists of two main modules: one provides support for algebraic
methods and the other is a set of APIs from different geometric
applications and formats to the prover.  

GeoProver produces detailed reports with steps that are taken in
proving process: transformation of input geometric problem to
algebraic form, invoking a specified algebraic-based theorem prover,
and presenting the result with time and space spent to prove the
theorem and with a list of NDG conditions obtained during proving
process, transformed to a readable, geometry form.

One of the main purposes of this Java implementation is integration
with various dynamic geometry tools (including GeoGebra) that
currently don't have support for proving geometry theorems. The
architecture of GeoProver enables easy integration with other systems
for interactive geometry and can be simply modified to accept various
input formats for conjectures.

\paragraph{Gr\"obner basis in Mathematica}

Mathematica is commercial software designed for technical
computing. It has support for Gr\"obner basis and can be used for
mechanical theorem proving. By $F = GroebnerBasis[\{poly_1\}, \{poly_2\},
  \ldots]$ are computed Gr\"obner basis for given list of polynomials
(representing premises of geometry conjecture). If
$PolynomialReduce[poly_{concl}, F, V][[2]]$ is zero ($V$ is set of
  variables), the geometry statement represented with polynomial
  $poly_{concl}$ is true.

\paragraph{Wu's method in Maple -- wsolve}

The implemented algorithm is based on Wu's method and his theorems
about the projection of quasi variety \cite{quasivarieties}. While
proving geometry conjecture, weakest non--degenerate conditions can be
obtained by computing the projection of a quasi variety. In fact, it
is possible to get the sufficient and necessary condition for a
geometric theorem to be false by computing the projection of a quasi
variety.

The method is easily invoked by key word \emph{wsolve(V, P, C)} where
$V$ is a list of variables, $P$ is a list of polynomials representing
premises of geometry statement and $C$ is polynomial representing
statement.


\section{Alegebrization of geometric relations in solid geometry}
\label{polynomials}
To apply algebraic methods we need to be able to represent various
geometric relations between solid geometry objects using polynomial
equations over their coordinates. In this section we are going to give
examples how this could be done, for the most common relations (a
richer set of relations can be represented using similar techniques,
and a detailed description is given in an online
appendix). % dodati url

There are different approaches for encoding relations in terms of
polynomial equations and the first question is what objects of 3d
geometry are basic. % popraviti
In the first approach, all objects are defined using points (e.g.,
lines are defined by two different points, and planes are defined as
three different, non-colinear points). The only variables used in
polynomials are coordinates of the points. In the other approach, all
types of objects are represented using their own coordinates (e.g., a
line is defined by the coordinates of its one point, and the
coordinates of its direction vector, and a plane is defined by the
coefficients of the plane equation -- the coordinates of its normal
vector, and its displacement wrt.~the origin). Polynomials include all
those coordinate variables. We shall show how to encode relations
using both of these approaches, and shall compare their efficiency.


\subsection{Basic notions used for constructing polynomials}
Most relations are expressed using the same set of notions which we
introduce here.

Each kind of object is represented by some tuple of parameters (we
shall see that these could have either symbolic, or numeric values).

Points will have three parameters, denoted by
$({[\_]}^x, {[\_]}^y, {[\_]}^z)$ representing their coordinates. Every
point is given either by its symbolic or numeric coordinates. To each
new point introduced point fresh symbolic coordinates are assigned.

Lines are represented differently, depending on the used approach.  In
the first approach a line is given by two different given points, and
a six-tuple of their coordinates. The first of the point of the line
$p$ will be denoted by $p_A$ and the second point by $p_B$.

In the other approach, a line is given by a given point $A$ and a
given vector $v$. Vector of the line named $p$ will be denoted by
$\overrightarrow{p_v}$ and the point of the line $p$ will be denoted by
$p_A$. Therefore, lines in the second approach will also have six
parameters denoted by
$({[\_]}^{v_x}, {[\_]}^{v_y}, {[\_]}^{v_z}, {[\_]}^{A_x},
{[\_]}^{A_y}, {[\_]}^{A_z})$,
which represent the line given by the equation:
$$x = k\cdot[\_]^{v_x} + {[\_]}^{A_x}\ \ y = k\cdot[\_]^{A_y} +
{[\_]}^{A_y}\ \ z = k\cdot[\_]^{v_z} + {[\_]}^{A_z}.$$
In the previous equation, $k$ denotes the line ratio, but this
information is not present in the line specification (which is a
six-tuple). Line ratio is going to be used in some polynomials that
describe constructions involving lines, but it is going to be
considered as a fresh symbolic variable.

Planes are also represented differently, depending on the used
approach. In the first approach a plane is given by three non-colinear
points, and a nine-tuple of their coordinates. The first point of the
plane $\pi$ will be denoted by $\pi_A$, the second one by $\pi_B$ and
the third one by $\pi_C$.

In the other approach, planes are determined by their normal vector
$v$ and an additional parameter $d$ (displacement wrt.~the
origin). Vector of the plane named $\pi$ will be denoted with
$\overrightarrow{\pi_v}$ and free parameter for the plane will be
denoted with $\pi_d$. Therefore, planes will have only four
parameters, denoted by
$({[\_]}^{v_x}, {[\_]}^{v_y}, {[\_]}^{v_z}, {[\_]}^{d})$, which
represent the plane given by the following equation:
$${[\_]}^{v_x}\cdot x + {[\_]}^{v_y}\cdot y + {[\_]}^{v_z}\cdot z +
{[\_]}^{d} = 0.$$ 

Vector determined by two points $A = (a^x, a^y, a^z)$ and
$B = (b^x, b^y, b^z)$ is
$\overrightarrow{AB} = (b^x- a^x, b^y - a^y, b^z - a^z)$. The standard
notions of scalar product, cross product and triple product can be
applied to vectors. Scalar product of vectors $v = (v^x, v^y, v^z)$
and $u = (u^x, u^y, u^z)$ is
$v\cdot u = v^x\cdot u^x+ v^y\cdot u^y + v^z\cdot u^z$, their vector
product is determined by the matrix:
$$ v\times u = \left|\begin{array}{ccc} \overrightarrow{i} & \overrightarrow{j} & \overrightarrow{k} \\ 
                       v^x& v^y & v^z \\
                       u^x& u^y & u^z \\
\end{array}\right|,$$
and their triple product with the vector $w = (w^x, w^y, w^z)$ is
equal to $v\cdot (u \times w)$, and is determined by the matrix:
$$\left|\begin{array}{ccc} v^x& v^y & v^z \\ u^x& u^y
  & u^z \\ w^x& w^y & w^z \\
\end{array}\right|.$$


\subsection{Representing relations between geometry objects}
In this section are given polynomials that arithmetically describe
relations over constructed objects (for example, two points coincide,
two lines are parallel, two planes are orthogonal).  Each kind of
relation introduces some polynomial constraints of the coordinates of
the objects that are involved, and are expressed differently depending
on the chosen approach.

Input parameters for given relation are parameters of all objects
involved in it. For example, for the relation \mbox{{\tt congruent}
  $A$ $B$ $C$ $D$} inputs are four points, $A$, $B$, $C$, and $D$, e.g
their symbolic coordinates: $(a^x, a^y, a^z)$, $(b^x, b^y, b^z)$,
$(c^x, c^y, c^z)$ and $(d^x, d^y, d^z)$.

\begin{description}
% ------------------------
\item[$\triangleright$] {\tt congruent} $A$ $B$ $C$ $D$

  {\em Description:} Two segments, $AB$ and $CD$ are
  congruent.

  %{\em Input:} Points $A$, $B$, $C$, and $D$.

{\em Polynomials:} \\
$\overrightarrow{AB} \cdot \overrightarrow{AB} = \overrightarrow{CD} \cdot \overrightarrow{CD}$. \\
Note that this gives the polynomial equation $poly = 0$, where

\begin{tabbing}
$poly = $ \= $({a^x} - {b^x})^2 + ({a^y} - {b^y})^2 + ({a^z} - {b^z})^2 -$ \\ 
          \> $({c^x} - {d^x})^2 - ({c^y} - {d^y})^2 - ({c^z} - {d^z})^2$
\end{tabbing}

{\em Explanation:} Squares of the distance between $A$ and $B$ must be
equal to the square of the distance between $C$ and $D$.

% ------------------------
\item[$\triangleright$] {\tt segments\_in\_ratio} $A$ $B$ $C$ $D$
  $m$ $n$ 

  {\em Description:} The lengths of segments $AB$ and
  $CD$ are in the given ratio $\frac{m}{n}$ i.e., that
  $\frac{|AB|}{|CD|} = \frac{m}{n}$.

{\em Polynomials:} \\
$n^2 \cdot \overrightarrow{AB} \cdot \overrightarrow{AB} = m^2 \cdot \overrightarrow{CD} \cdot \overrightarrow{CD}$. \\


{\em Explanation:} The squares of distances between $A$ and $B$ and
between $C$ and $D$ must be in the ratio $\frac{m^2}{n^2}$. Note that
this reduces to congruence when $m = n$.

% ------------------------
\item[$\triangleright$] {\tt is\_midpoint} $M$ $A$ $B$

{\em Description:} Checks weather point $M$ is a midpoint of the
segment determined with points $A$ and $B$.

{\em Polynomials:} $\overrightarrow{MA} = \overrightarrow{MB}$.  Note
that this yields three different polynomials $poly_1 = 0$,
$poly_2 = 0$, $poly_3 = 0$:

\begin{tabbing}
$poly_1 = $ \= $2m^x - a^x - b^x$ \\ 
$poly_2 = $ \= $2m^y - a^y - b^y$ \\ 
$poly_3 = $ \= $2m^z - a^z - b^z$
\end{tabbing}

% ------------------------
\item[$\triangleright$] {\tt point\_segment\_ratio} $M$ $A$ $B$ $p$ $q$

{\em Description:} Checks weather the point $M$ divides segment
determined with points $A$ and $B$ in ratio determined with $p$ and
$q$, e.g. $\frac{|MA|}{|MB|} = \frac{p}{q}$.

{\em Polynomials:} three polynomials derived from
$q\cdot \overrightarrow{MA} = p\cdot \overrightarrow{MB}$.

{\em Explanation:} Note that {\tt is\_midpoint $M$ $A$ $B$} can be
also written using this rule as {\tt point\_segment\_ratio} $M$ $A$
$B$ $1$ $1$.

% ------------------------
\item[$\triangleright$] {\tt equal\_points} $A$ $B$

{\em Description:} Checks weather two points $A$ and $B$ have the same
coordinates.

{\em Polynomials:} three polynomials from the expression
$\overrightarrow{AB} = 0$.

% ------------------------
\item[$\triangleright$] {\tt orthogonal\_4points} $A$ $B$ $C$ $D$

{\em Description:} Checks weather line determined with points $A$
and $B$ is orthogonal on the line determined with points $C$ and
$D$.

{\em Polynomials:} $\overrightarrow{AB} \cdot \overrightarrow{CD} = 0$

% ------------------------
\item[$\triangleright$] {\tt orthogonal\_lines} $p$ $q$
% Dodati pricu o proveri mimoilaznosti

{\em Description:} Two lines, $p$ and $q$ are orthogonal.

{\em Polynomials:} If the first approach is used, then the lines are
given by the points $(p_A, p_B)$ and $(q_A, q_B)$, so this reduces to
the previous case and the polynomial is
$\overrightarrow{p_Ap_B} \cdot \overrightarrow{q_Aq_B} = 0$. If the
second approach is used, the direction vectors of the lines are given
as the input, and the polynomial is
$\overrightarrow{p_v} \cdot \overrightarrow{q_v} = 0.$

% ------------------------
\item[$\triangleright$] {\tt incident} $A$ $p$

{\em Description:} Checks weather the point $A$ belongs to the line
$p$.

{\em Polynomials:} If the first approach is used, then the three
polynomials are derived from
$\overrightarrow{p_Ap_B} \times \overrightarrow{Ap_A} = 0$. If the
second approach is used, then the three polynomials are derived from
$\overrightarrow{p_v} \times \overrightarrow{Ap_A} = 0.$

% ------------------------
\item[$\triangleright$] {\tt parallel\_lines} $p$ $q$

{\em Description:} Checks weather two lines, $p$ and $q$ are parallel.

{\em Polynomials:} The three polynomials are derived from
$\overrightarrow{p_Ap_B} \times \overrightarrow{q_Aq_B}$ or
$\overrightarrow{p_v} \times \overrightarrow{q_v}$ depending on the
approach.

% ------------------------
\item[$\triangleright$] {\tt parallel\_planes} $\alpha$ $\beta$

{\em Description:} Checks weather two planes $\alpha$ and $\beta$ are
parallel.

{\em Polynomials:} If the second approach is used, the three
polynomials are derived from
$\overrightarrow{\alpha_v} \times \overrightarrow{\beta_v} = 0.$

% Videti da li je moguce dobiti malo manje polinome ako se prvi pristup
% ne primenjuje na isti nacin kao drugi.

% Ovde raspisati da bi se videlo kako su ovi polinomi orgromni i ruzni

$$\overrightarrow{\beta_A\beta_B}\cdot \overrightarrow{\alpha_A\alpha_C} \times \overrightarrow{\alpha_B\alpha_A} = 0$$
$$\overrightarrow{\beta_A\beta_C}\cdot \overrightarrow{\alpha_A\alpha_C} \times \overrightarrow{\alpha_A\alpha_B} = 0$$

$$(\overrightarrow{\beta_A\beta_C}\times\overrightarrow{\beta_A\beta_B}) \times (\overrightarrow{\alpha_A\alpha_C} \times \overrightarrow{\alpha_A\alpha_B}) = 0$$


% ------------------------
\item[$\triangleright$] {\tt orthogonal\_planes} $\alpha$ $\beta$

{\em Description:} Checks weather two planes, $\alpha$ and $\beta$ are
orthogonal.

{\em Polynomials:} 
If the first approach is used ....
If the second approach is used, the polynomial
$\overrightarrow{\alpha_v} \cdot \overrightarrow{\beta_v} = 0$.

% ------------------------
\item[$\triangleright$] {\tt point\_in\_plane} $A$ $\pi$

{\em Description:} Checks weather the point $A$ belongs to the plane
$\pi$.

{\em Polynomials:} If the first approach is used, then the polynomial
$$\overrightarrow{\pi_AA}\cdot (\overrightarrow{\pi_A\pi_B} \times \overrightarrow{\pi_A\pi_C}) = 0.$$
If the second approach is used, then the polynomial
$\overrightarrow{\pi_v} \cdot \overrightarrow{A} + \pi^{d} = 0$.

% ------------------------
\item[$\triangleright$] {\tt parallel\_line\_plane} $p$ $\alpha$

{\em Description:} Checks weather the line $p$ and plane $\alpha$ are
parallel.

{\em Polynomials:} 
If the first approach is used ...
If the second approach is used, then polynomial
$\overrightarrow{p_v} \cdot \overrightarrow{\alpha_v} = 0$.

{\em Note:} This statement also holds if line belongs to the plane.

% ------------------------
\item[$\triangleright$] {\tt orthogonal\_line\_plane} $p$ $\alpha$

{\em Description:} Checks weather the line $p$ and plane $\alpha$ are
orthogonal.

{\em Polynomials:} If the first approach is used ...

If the second approach is used the three polynomials
$\overrightarrow{p_v} \times \overrightarrow{\alpha_v} = 0$.

% Dodati pravila za uglove 
% Dodati pravila za pravilne objekte (tetraedar npr.)
\end{description}

\section{Simplification of polynomials (needed for Wu's method)}
\label{simplification}

The first problem when applying GeoProver onto generated sets of
polynomials were reaching time and space limit. The reason for this is
large number of variables and polynomials that were
created. Furthermore, polynomials in solid geometry are more complex
than corresponding polynomials in plane geometry. Thus, there was a
need to simplify our system.

We used approach proposed by Harrison, without lost of generality
reasoning --- "wlog" \cite{harrison} by making a convenient choice of
coordinate system since different choice of coordinate system could
make algebraic calculations much less robust. 

For three independently given points $A$, $B$ and $C$, it is possible
to choose their coordinated in such a manner that $A(0, 0, 0)$ is in
the origin, $B(0, 0, b^z)$ is on the $z$-axis and $C(0, c^y, c^z)$
lies in plane $yOz$. With this choice of coordinates, the number of
variables is reduced by six and the corresponding zeroes simplify
polynomials. This approach is commonly used in algebraic methods. It
does not affect the generality of statement and justification for its
usage comes from the fact that rotations and translations can be used
to transform points into its "canonical" position. Translation and
rotation are isometries which means that they preserve distance, and
then also the other geometric relations such as incidence,
orthogonality, size of an angle and so on.

We extended this approach and applied similar reasoning on lines and
plains. In our list of constructions there exist one construction for
arbitrary given line, {\tt line} $l$. With this construction rule, six
variables are generated, vector of the line $(l^{v_1}, l^{v_2},
l^{v_3})$ and the coordinates some point on the line $(p^x, p^y,
p^z)$. However, it is possible to choose a more convenient
coordinates, for vector $(0, 0, l^{v_3})$ and for point $(0, 0, 0)$,
which means that as arbitrary line is chosen $z$-axis. Similarly goes
for plains, and for arbitrary given plane, {\tt plane} $\pi$, it is
possible to choose coordinates $\pi (0, 0, p^{v_3}, 0)$, which means
that arbitrary chosen plain is plain $xOy$.

Without applying this method, even the simplest statements could not
be proved. Choosing convenient coordinates greatly improved simplicity
of the system of polynomials and, by doing so, made the program more
efficient. However, some polynomials become redundant, as they become
$0$ and we just deleted them from the system. Also, some polynomials
become a polynomial of one variable on some degree and they were also
deleted from the system and the value of the corresponding variable
was set to zero. Although, it was not necessary to delete these
polynomials, the system was simplified a bit more when they were
deleted.

To simplify constructions even further, we choose appropriate
coordinates for figures as we explained in \ref{objectconstruction}.

Furthermore, polynomials whose structure is $x_i - x_j$ or $x_i + x_j$
(assume $j < i$) were also deleted from the system, each appearance of
$x_i$ was replaced with $x_j$, e.g. with $-x_j$, and $x_i$ was deleted
from the set of variables. This simplification step was not very
crucial, and most of the tested statements could be proved without
adding this simplification. However, for some complex statements, it
was very important to reduce the size of sets of polynomials
(otherwise, space or time limit is reached) and this is a step into
that direction.


\subsection{Construction rules}


Each construction starts from some given (free) objects and proceeds
by introducing new (dependent) objects. We will allow introducing only
free points (and all lines, planes, and solids are going to be
constructed started from a given set of free points).

Free points are determined by their fresh, symbolic coordinates, and
are not restricted by any polynomials. Constructed objects are
determined again by introducing their fresh symbolic parameters, but
also polynomials that describe connections between those parameters
and the symbolic parameters of the objects that were previously
constructed. We shall denote the parameters of free points by
upper-case letters, and the parameters of dependent points by
lower-case letters. For example, the free point $A$ will have symbolic
coordinates $({A}^x, {A}^y, {A}^z)$, and a constructed line $l$ will
have symbolic parameters (${l}^{v_x}, {l}^{v_y}, {l}^{v_z}, {l}^{p_x},
{l}^{p_y}, {l}^{p_z}$).


\paragraph{Constructions of points}

Next we shall describe some example construction steps and their
corresponding polynomials. Each construction step yields a new
point. Its name will be given by the first argument of the
construction step (e.g., {\tt midpoint} $M$ $A$ $B$ constructs the
point $M$ using given points $A$ $B$). In each such construction step,
it is assumed that the symbolic parameters of given objects are
already known, and the symbolic coordinates of the resulting point are
introduced (as fresh variables).

\begin{description}

% ------------------------
\item[$\triangleright$] {\tt point} $A$ \\
  {\em Description:} This step introduces a free point $A$.\\
  {\em Input:} none\\
  {\em Objects and parameters:} A point $A$ with the three fresh parameters
  $(A_x, A_y, A_z)$ \\
  {\em Polynomials:} no polynomials are generated. \\

% ------------------------
\item[$\triangleright$] {\tt point\_ratio} $M$ $A$ $B$ $p$ $q$ 

  {\em Description:} Construction of a point that divides a segment in
  a given ratio, i.e., for two given points $A$ and $B$ (these could
  be either free or dependent points), and two integer numbers $p$ and
  $q$, this step constructs the point $M$ so that
  $\frac{|AM|}{|BM|} = \frac{p}{q}$ holds.

  \marginpar{mislim da nema potrebe svuda stavljati "Input" i
    "Output". Mozda u prvom primeru objasniti, ali deluje redudatno da
    se svuda stavlja jer ce citaocu iz opisa biti jasno sta su dati, a
    sta novi objekti, a ranije je receno da kada kazemo dat je novi
    objekat da to znaci date su njegove koordinate. Ili dat je novi
    objekat, znaci uvode se nove koordinate za taj objekat. Mislim da
    je suvisno svuda pisati "ulaz", "izlaz".}

  {\em Input:} Two given points $A$ and $B$ with symbolic
  coordinates $(a^x, a^y, a^z)$ and $(b^x, b^y, b^z)$.

  {\em Objects and parameters: } The point $M$ with three fresh
  symbolic coordinates $(m^x, m^y, m^z)$.

\begin{tabbing}
{\em Polynomials:} \= $poly_1 = p\cdot m^x + q\cdot m^x - p\cdot b^x - q\cdot a^x$ \\
                   \> $poly_2 = p\cdot m^y + q\cdot m^y - p\cdot b^y - q\cdot a^y$ \\
                   \> $poly_3 = p\cdot m^z + q\cdot m^z - p\cdot b^z - q\cdot a^z$ 
\end{tabbing}

{\em Explanation:} {\tt point\_ratio} $M$ $A$ $B$ $p$ $q$ = {\tt point\_segment\_ratio} $M$ $A$ $B$ $p$ $q$

{\em Note:} Note that {\tt midpoint} is a special case of this
construction, e.g. it is possible to determine midpoint in the
following way \mbox{{\tt point\_ratio} $M$ $A$ $B$ $1$ $1$}. However,
we also support the \mbox{{\tt midpoint} $M$ $A$ $B$} construction.

% ------------------------
\item[$\triangleright$] {\tt intersection\_lines} $A$ $l_1$ $l_2$

  {\em Description:} This construction introduces the point $A$ that is
  the intersection of the two given lines $l_1$ and $l_2$.

  {\em Input:} The two given lines $l_1$ and $l_2$ with parameters
  $l_1(l_1^{v_x}, l_1^{v_y}, l_1^{v_z}, l_1^{p_x}, l_1^{p_y},
  l_1^{p_z})$ and $l_2(l_2^{v_x}, l_2^{v_y}, l_2^{v_z}, l_2^{p_x}
  l_2^{p_y}, l_2^{p_z})$.

  {\em Objects and parameters:} The point $A$ with three fresh
  symbolic coordinates $(a^x, a^y, a^z)$.

  {\em Polynomials:} We generate polynomials using rule:
  {\tt intersection\_lines} $A$ $l_1$ $l_2$ = {\tt incident} $A$ $l_1$ and {\tt incident} $A$ $l_2$

\begin{tabbing}
{\em Polynomials for Wu's method:} \= $poly_1 = a^x - k_1\cdot l_1^{v_x} - l_1^{p_x}$ \\
                   \> $poly_2 = a^y - k_1\cdot l_1^{v_y} - l_1^{p_y}$ \\
                   \> $poly_3 = a^z - k_1\cdot l_1^{v_z} - l_1^{p_z}$ \\
                   \> $poly_4 = a^x - k_2\cdot l_2^{v_x} - l_2^{p_x}$ \\
                   \> $poly_5 = a^y - k_2\cdot l_2^{v_y} - l_2^{p_y}$ \\
                   \> $poly_6 = a^z - k_2\cdot l_2^{v_z} - l_2^{p_z}$
\end{tabbing}

{\em Explanation:} 
Since the point $a$ should belong to both lines $l_1$ (given by the
point $\overrightarrow{l_1^p}$ and the vector
$\overrightarrow{l_1^v}$) and $l_2$ (given by the point
$\overrightarrow{l_2^p}$ and the vector $\overrightarrow{l_2^v}$) it
must satisfy their parametric equations, i.e., it must hold that
$\overrightarrow{A} = \overrightarrow{l_1^p} + k_1 \cdot
\overrightarrow{l_1^v}$
and
$\overrightarrow{A} = \overrightarrow{l_2^p} + k_2 \cdot
\overrightarrow{l_2^v}$.

This is the most interesting case of point construction. In
tree--dimensional space not all lines have intersection, some lines
can be parallel, and others can be skew lines. Note that the
polynomials use two unknown variables $k_1$ and $k_2$ that
respectively represent line ratios for lines $l_1$ and
$l_2$. Therefore, the total number of unknown variables is five, but
there are six polynomials that this construction generates. If all six
polynomials are included in the construction set, then it will not be
possible to make triangular system since there will be more
polynomials than unknown variables.\marginpar{Govori se o postupku
  resavanja o kom nije jos bilo reci.} Five polynomials can be used to
determine the solution of the system of equations and the remaining
polynomial should be used to justify the solution. Any of the
polynomials could be excluded and in our solution we always exclude
$p_6$. If all five chosen polynomials have zero value for the
calculated coordinates, then also the remaining polynomial must be
zero (or otherwise, the lines will not always intersect). If the
excluded polynomial is not considered, that could lead into wrong
constructions and conclusions since now it is possible to (falsely)
determine the intersection point of parallel or skew lines. So,
instead of just deleting this last polynomial, is should be put it in
the conclusion set.

% ------------------------
\item[$\triangleright$] {\tt intersection\_4points} $M$ $A$ $B$ $C$ $D$

  {\em Description:} For four given point $A$, $B$, $C$ and $D$
  determine the point $M$ which is intersection of line determined
  with points $A$ and $B$ and line determined with points $C$ and
  $D$. This is very similar to the construction we described above.

  {\em Input:} Four points $A$, $B$, $C$ and $D$ with symbolic
  coordinates $(a^x, a^y, a^z)$, $(b^x, b^y, b^z)$, $(c^x, c^y, c^z)$
  and $(d^x, d^y, d^z)$.

  {\em Objects and parameters:} The point $M$ with three fresh
  symbolic coordinates $(m^x, m^y, m^z)$.

\begin{tabbing}
{\em Polynomials:} \= $poly_1 = m^x - k^x\cdot b^x + k_1\cdot a^x - a^x$ \\
                   \> $poly_2 = m^y - k^x\cdot b^y + k_1\cdot a^y - a^y$ \\
                   \> $poly_3 = m^z - k^x\cdot b^z + k_1\cdot a^z - a^z$ \\
                   \> $poly_4 = m^x - k^y\cdot d^x + k_2\cdot c^x - c^x$ \\
                   \> $poly_5 = m^y - k^y\cdot d^y + k_2\cdot c^y - c^y$ \\
                   \> $poly_6 = m^z - k^y\cdot d^z + k_2\cdot c^z - c^z$
\end{tabbing}

{\em Explanation:} The polynomials were generated using the following
line equations: \\ $x = k_1\cdot (b^x - a^x) + a^x \ \ \ \ y =
k_1\cdot (b^y - a^y) + a^y \ \ \ \ z = k_1\cdot (b^z - a^z) + a^z$ ---
line determined with points $A$ and $B$ \\ $x = k_2\cdot (d^x - c^x) +
c^x \ \ \ \ y = k_2\cdot (d^y - c^y) + c^y \ \ \ \ z = k_2\cdot (d^z -
c^z) + c^z$ --- line determined with points $C$ and $D$.



This construction could be performed by explicitly constructing lines
$AB$ and $CD$ and then constructing their intersection using the
previous rules. However, it turns out that this would introduce many
new symbolic variables (parameters of the two lines), the system would
become more complex, and the algebraic solver would have to work with
a much more difficult task. Therefore, it is much more efficient to
have this construction as an elementary step, especially, having in
mind that this step is used very often in classic geometric problems. 

Note that again there are six polynomials and five unknown
variables. This problem is resolved in the same manner as above by
putting polynomial $poly_6$ in the conclusion set. The reason is the
same as in the previous construction rule and again it is possible not
to have intersection since lines can be parallel or skew.
\end{description}


\paragraph{Constructions of lines}
Although the construction of a free line could be easily introduced,
for simplicity, we have chosen to support only free points (and a free
line can be constructed as a line trough two free points).

\begin{description}
% ------------------------
\item[$\triangleright$] {\tt line\_points} $l$ $A$ $B$

  {\em Description} For two given points $A$ and $B$ determines the
  line $l$ that goes through these points.


  {\em Input:} Two points $A$ and $B$ with symbolic coordinates $(a^x,
  a^y, a^z)$ and $(b^x, b^y, b^z)$.

  {\em Objects and parameters:} The line $l$ with parameters
  $(l^{v_x}, l^{v_y}, l^{v_z}, a^x, a^y, a^z)$ (new, fresh parameters
  $(l^{v_x}, l^{v_y}, l^{v_z})$ are parameters of line vector).

\begin{tabbing}
{\em Polynomials:} \= $poly_1 = l^{v_x} - a^x + b^x$ \\
                   \> $poly_2 = l^{v_y} - a^y + b^y$ \\
                   \> $poly_3 = l^{v_z} - a^z + b^z$ \\
                   \> $\overrightarrow{l} = \overrightarrow{AB}$ \\
                   \> {\tt line\_points} $l$ $A$ $B$ = {\tt incident} $A$ $l$ and {\tt incident} $B$ $l$
\end{tabbing}

{\em Explanation:} Each line is determined by a set of 6 parameters,
but note that in this construction we have chosen to introduce only
three. This is due to the fact that the parameters for the point on
the line can be exactly the coordinates of the point $A$ (or $B$),
that are already introduced symbolic variables, so there is no need to
introduce new variables and polynomials (if the parameters $l^{p_x}$,
$l^{p_y}$ and $l^{p_z}$ were introduced, they would need to satisfy
the trivial polynomials $l^{p_x} - a_x$, $l^{p_y}-a_y$ and $l^{p_z} -
a_z$). The parameters $(l^{v_x}, l^{v_y}, l^{v_z})$ that correspond to
the vector of the line must satisfy $\overrightarrow{v} =
\overrightarrow{BA}$, and the polynomials are generated from this
equation (when written in symbolic coordinates).

% ------------------------
\item[$\triangleright$] {\tt line\_orth\_plane} $l$ $\pi$ $A$ \\

{\em Description:} For a given plane $\pi$ and a point $A$ determines
line $l$ witch is orhogonal on the given plane $\pi$ and incides with
the point $A$.

{\em Input:} Plane $\pi$ with symbolic parameters $(\pi^{v_x},
\pi^{v_y}, \pi^{v_z}, \pi^{d})$ and point $A$ with symbolic
coordinates $(a^x, a^y, a^z)$.

{\em Objects and parameters:} Line $l$ with parameters $(\pi^{v_x},
\pi^{v_y}, \pi^{v_z}, \pi^{d}, a^x, a^y, a^z)$ (no new parameters are
generated).

{\em Polynomials:} No polynomials are generated.
                   {\tt line\_orth\_plane} $l$ $\pi$ $A$ = {\tt incident} $A$ $l$ and {\tt parallel\_line\_plane} $l$ $\pi$

{\em Explanation:} Both the parameters for the point on the line and
the vector of the line can be represented by the already introduced
symbolic coordinates --- the point can be the point $A$ and the line
vector could be equal to the normal vector of the plane. Therefore, it
was not necessary to add neither new symbolic parameters, nor the new
polynomials.
\end{description}

The construction of the line trough a given point that is parallel to
the given line is performed similarly (the parameters of the point and
the parameters of the vector of the given line can be reused, and no
new parameters and polynomials need to be introduced).


\paragraph{Constructions of the plane}
Again, for simplicity, we have chosen to support only free points (and
a free plane can be constructed as a plane trough three free points).

\begin{description}
% ------------------------
\item[$\triangleright$] {\tt plane\_points} $\pi$ $A$ $B$ $C$

{\em Description:} For three given points $A$, $B$ and $C$ determines
the plane $\pi$ containing all of them.

{\em Input:} Tree point $A$, $B$ and $C$ with symbolic coordinates
$(a^x, a^y, a^z)$, $(b^x, b^y, b^z)$ and $(c^x, c^y, c^z)$.

{\em Objects and parameters:} Plane $\pi$ with fresh parameters
$(\pi^{v_x}, \pi^{v_y}, \pi^{v_z}, \pi^{d})$.

\marginpar{ovo nije sasvim tacno, jer ovde imamo tri jednacine, a 4 nepoznate}
\begin{tabbing}
{\em Polynomials:} \= $poly_1 = \pi^{v_x} - b^y\cdot c^z + b^y\cdot a^z + a^y\cdot c^z + b^z\cdot c^y - b^z\cdot a^y - a^z\cdot c^y$ \\
                   \> $poly_2 = \pi^{v_y} - b^x\cdot c^z - b^x\cdot a^z - a^x\cdot c^z - b^z\cdot c^x + b^z\cdot a^x + a^z\cdot c^x$ \\
                   \> $poly_3 = \pi^{v_z} - b^x\cdot c^y + b^x\cdot a^y + a^x\cdot c^y + b^y\cdot c^x - b^y\cdot a^x - a^y\cdot c^x$ \\
                   \> $poly_4 = \pi^{d} + \pi^{v_x}\cdot a^x + \pi^{v_y}\cdot a^y + \pi^{v_z}\cdot a^z$ \\
                   \> {\tt plane\_points} $\pi$ $A$ $B$ $C$ = \= {\tt point\_in\_plane} $A$ $\pi$ and \\
                                                              \>\> {\tt point\_in\_plane} $B$ $\pi$ and \\
                                                              \>\> {\tt point\_in\_plane} $C$ $\pi$ \\
\end{tabbing}

{\em Explanation:} The normal vector of the line can be obtained as a
vector product of the vectors $\overrightarrow{AB}$ and
$\overrightarrow{AC}$. Since $(b^x - a^x, b^y - a^y, b^z - a^z)$ and
$(c^x - a^x, c^y - a^y, c^z - a^z)$ are vectors of two lines
determined with points $A$ and $B$, and points $C$ and $D$, the first
three polynomials are derived using the formula:
$$(\pi^{v_x}, \pi^{v_y}, \pi^{v_z}) = \begin{array}{ccc} i & j & k \\ b^x
  - a^x & b^y - a^y & b^z - a^z \\ c^x - a^x & c^y - a^y & c^z - a^z
\end{array}$$

The last parameter of the plane $\pi^{d}$ is obtained from the
condition that the point $A$ is in the plane $\pi$, and satisfies its
equation. Note that this equation uses the symbolic parameters
$\pi^{v_x}$, $\pi^{v_y}$ and $\pi^{v_z}$ that are introduced in this
construction step.
\end{description}

Very similar methods can be used to construct the plane that contains
the given line and is orthogonal or is parallel to the given plane.

\paragraph{Solid construction}
\label{objectconstruction}
\marginpar{Da li uvesti izometrijske transformacije?}

Many exercises in solid geometry have objects and usually start with a
sentence of the form {\em "In a given cube..."}, {\em "In a given
  pyramid..."}, {\em "In a given prism..."} and so on. The first
approach was to construct such solids using construction rules
presented above. This means that there were many rules (around 20)
constructing only one cube and thus many new variables and polynomials
which made polynomial systems very complex. Also, some important
construction steps could be forgotten. Therefore, it is very natural
to introduce solid constructions as elementary construction steps
(instead of ten or twenty rules that construct their points, lines and
planes). Such construction steps automatically introduce only the
points of the solid, whereas lines (edges) and planes (faces) can be
introduced by the previous construction rules only when necessary.


We decide to support only the construction of solids placed in some
{\em canonical} positions --- for example, when constructing a cube,
one vertex is placed in the origin, and the other three are placed on
the axes ($x$-axis, $y$-axis and $z$-axis). However, this approach has
a drawback. For example, it is not possible to construct more than one
different cube using the elementary cube construction step. The points
of other cubes that are not in the canonical position can be
constructed using point construction steps or by applying isometric
transformations to the canonical cube. On the other hand, in most
geometric problems encountered in education there is usually only one
free solid, and without loss of generality it can be assumed that it
is in the canonical position. When other solids are introduced in the
problem text, they are usually dependent, so their points (vertices)
have to be constructed. Therefore, supporting only canonical solid
constructions was not the obstacle for the problems encountered in
most exercises.

\begin{description}
% ------------------------
\item[$\triangleright$] {\tt make\_cube} \\
  {\em Description:} Construct the cube in the canonical position, with the edge length equal to $1$.\\
  {\em Input}: none\\
  {\em Objects and parameters:} Points $A(0, 0, 0)$, $B(1, 0, 0)$,
  $C(1, 1, 0)$, $D(0, 1, 0)$, $A_1(0, 0, 1)$, $B_1(1, 0, 1)$,
  $C_1(1, 1, 1)$ and $D(0, 1, 1)$.\\
  {\em Polynomials:} No polynomials are generated. \\
  {\em Explanation:} Since the cube is in the canonical position, no
  symbolic variables are introduced.

% ------------------------
\item[$\triangleright$] {\tt make\_tetrahedron} \\
  {\em Description: }Construct the tetrahedron in the canonical
  position.\\
  {\em Input: } none\\
  {\em Objects and parameters:} The vertices of the tetrahedron have
  the coordinates $A(0, 0, 0)$, $B(1, 0, 0)$, $C(c^x, c^y, 0)$ and
  $D(c^x, d^y, d^z)$, with the four fresh parameters.
\begin{tabbing}
{\em Polynomials:} \= $poly_1 = 2\cdot c^x - 1$ \\
                   \> $poly_2 = 2\cdot {c^y}^2 - 3$ \\
                   \> $poly_3 = 3\cdot d^y - c^y$ \\
                   \> $poly_4 = 3\cdot {d^z}^2 - 2$
\end{tabbing}

{\em Explanation:} $c^x = \frac{1}{2}$, $c^y = \frac{\sqrt{3}}{2}$,
$d^y = \frac{\sqrt{3}}{6} = \frac{c^y}{3}$, $d^z =
\frac{\sqrt{2}}{\sqrt{3}}$. Note that all our objects always have
either symbolic or integer parameters and polynomials must always have
integer coefficients, so irrational values (and even fractions) must
be introduced using polynomials.

% ------------------------
\item[$\triangleright$] {\tt make\_pyramid\_4side}

  {\em Description:} Construct a regular pyramid in the canonical
  position -- the base is a unit square in the $xOy$ plane, its
  lateral edges are equal in length, and the height is not
  constrained.
  
  {\em Input:} none

  {\em Objects and parameters:} Points $A(0, 0, 0)$, $B(1, 0, 0)$,
  $C(1, 1, 0)$, $D(0, 1, 0)$ and $S(s^x, s^y, s^z)$, for three fresh
  parameters $s^x$, $s^y$ and $s^z$.
\begin{tabbing}
{\em Polynomials:} \= $poly_1 = 2\cdot s^x - 1$ \\
                   \> $poly_2 = 2\cdot s^y - 1$
\end{tabbing}

{\em Explanation:} The projection of the apes is $(s^x, s^y, 0)$ and
it lies in the middle of the square, so $s^x = s^y =
\frac{1}{2}$. Note that $s^z$ is not constrained.

% ------------------------
\item[$\triangleright$] {\tt make\_square}

  {\em Description:} Construct a square in the canonical position --
  the base is a unit square in the $xOy$ plane, its lateral edges are
  equal in length, and the height is not constrained.
  
  {\em Input:} none

  {\em Objects and parameters:} Points $A(0, 0, 0)$, $B(1, 0, 0)$,
  $C(1, 1, 0)$, $D(0, 1, 0)$.

\begin{tabbing}
{\em Polynomials:} no new polynomials are created.
\end{tabbing}

% ------------------------
\item[$\triangleright$] {\tt make\_point\_on\_line} $A$ $l$

  {\em Description:} Construct a point that belongs to the given line.
  
  {\em Input:} Line $l$ with symbolic parameters with parameters
  $l(l^{v_x}, l^{v_y}, l^{v_z}, l^{p_x}, l^{p_y}, l^{p_z})$.

  {\em Objects and parameters:} Point $A(a^x, a^y, a^z)$, for three
  fresh parameters $a^x$, $a^y$ and $a^z$.

\begin{tabbing}
{\em Polynomials:} \= $poly_1 = l^{p_x}\cdot l^{v_y} - a^x\cdot l^{v_y} - l^{p_y}\cdot l^{v_x} + a^y\cdot l^{v_x}$ \\
                   \> $poly_2 = l^{p_x}\cdot l^{v_z} - a^x\cdot l^{v_z} - l^{p_z}\cdot l^{v_x} + a^z\cdot l^{v_x}$ \\
                   \> $poly_3 = l^{p_y}\cdot l^{v_z} - a^y\cdot l^{v_z} - l^{p_z}\cdot l^{v_y} + a^z\cdot l^{v_y}$ \\
                   \> {\tt make\_point\_on\_line $A$ $l$} = {\tt incident} $A$ $l$
\end{tabbing}

% ------------------------
\item[$\triangleright$] {\tt make\_point\_in\_plane} $A$ $\pi$ 

  {\em Description:} Construct a point that belongs to the given
  plane.
  
  {\em Input:} Plane $\pi$ with fresh parameters $(\pi^{v_x},
  \pi^{v_y}, \pi^{v_z}, \pi^{d})$.

  {\em Objects and parameters:} Point $A(a^x, a^y, a^z)$, for three
  fresh parameters $a^x$, $a^y$ and $a^z$.

\begin{tabbing}
{\em Polynomials:} \= $poly = \overrightarrow{\pi} \circ \overrightarrow{A} + \pi^{d}$ \\
                   \> {\tt make\_point\_in\_plane} $A$ $\pi$  = {\tt point\_in\_plane} $A$ $\pi$

\end{tabbing}

% ------------------------
\item[$\triangleright$] {\tt translate\_z} $A$ $O$ $q$

  {\em Description:} Construct a point given by translating another
  point by some parameter along $z$-axis.
  
  {\em Input:} Point $O$ with symbolic coordinates $(o^x, o^y, o^z)$
  and parameter $q$.

  {\em Objects and parameters:} Point $A(o^x, o^y, a^z)$, for one
  fresh parameter, $a^z$.

\begin{tabbing}
{\em Polynomials:} \= $poly = a^z - q - o^z$ \\
                   $\overrightarrow{AO} = (0, 0, q)$
\end{tabbing}
% ------------------------
\item[$\triangleright$] {\tt equilateral\_triangle} $A$ $B$ $C$

  {\em Description:} Construct a equilateral triangle in the canonical
  position -- it is in $xOy$ plane, one point is in origin, and
  another point is on $x$-axis.
  
  {\em Input:} none

  {\em Objects and parameters:} Points $A(0, 0, 0)$, $B(1, 0, 0)$,
  $C(c^x, c^y, 0)$, for two fresh parameters $c^x$ and $c^y$.

\begin{tabbing}
{\em Polynomials:} \= $poly_1 = 2\cdot c^x - 1$ \\
                   \> $poly_2 = 4\cdot c^y - 3$
\end{tabbing}


% ------------------------
\item[$\triangleright$] {\tt regular\_hexagon} $A_1$ $A_2$ $A_3$ $A_4$ $A_5$ $A_6$

  {\em Description:} Construct a regular hexagon in the canonical
  position -- it is in $xOy$ plane, one point is in origin, and
  another point is on $x$-axis. 
  
  {\em Input:} none

  {\em Objects and parameters:} Points $A_1(0, 0, 0)$, $A_2(1, 0, 0)$,
  $A_3(a_3^x, a_3^y, 0)$, $A_4(1, a_4^y, 0)$, $A_5(0, a_4^y, 0)$ and
  $A_6(a_6^x, a_3^y, 0)$, for four fresh parameters $a_3^x$, $a_3^y$,
  $a_4^y$, and $a_6^x$.

\begin{tabbing}
{\em Polynomials:} \= $poly_1 = 2\cdot a_3^x - 3$ \\
                   \> $poly_2 = 4(a_3^y)^2 - 3$ \\
                   \> $poly_3 = a_4^y - 3$ \\
                   \> $poly_4 = 2a_6^x - 1$
\end{tabbing}

\end{description}

\section{Examples}


\section{Non-degeneracy conditions}

\section{Conclusion and future work}

To the best of our knowledge there does not exist automated theorem
prover for solid geometry. Automated methods (such as Wu's method or
Gr\"obner basis method) can prove solid geometry statements, but using
these methods requires transforming geometry statements into
polynomial equations.

We studied two different algebraization methods for geometry
statements. The first approach introduces only coordinates of the
points involved in the construction, whereas the second approach
introduces coordinates of the objects involved in the geometry
statements such as lines, planes and objects. By comparing these two
approaches on the same set of problems, we concluded that algebraic
provers are more efficient if their input is obtained using second
approach. While testing, we noticed that the significant property is
the complexity of the polynomials in the set. The simpler polynomials
leads to better efficiency. It is especially important (if possible)
to avoid quadratic polynomials --- better approach is to substitute
complex quadratic polynomial with two or more simpler ones. Though the
smaller number of polynomials and the smaller number of variables also
help in better productivity, these two properties are not the key ones
and their impact is less obvious. Using second approach, obtained
polynomials are less complex and that is the reason why the second
approach was more convenient for algebraic provers.

During this research, we also examined different implementations of
algebraic provers. GeoProver's Wu's method required to be possible to
transform polynomials in the premises into triangular system. However,
this property is hard to secure and special care should be taken for
intersection of two lines. Thus, we felt more comfortable using the
other two provers, Maple's Wu's method and Mathematica's Gr\"oebner
basis method.



\bibliographystyle{elsarticle-harv}
% Include the ".bib" file (generated by bibtex) right here.
\bibliography{algmethods3d}

\end{document}
