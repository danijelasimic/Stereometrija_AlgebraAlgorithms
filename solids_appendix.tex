\documentclass{article}

\usepackage{fullpage}

\usepackage{amsmath}
\usepackage{amsthm}
\usepackage{amssymb}
\usepackage{amsfonts}
\usepackage{comment}
\usepackage{url}
\usepackage{tikz}
\usepackage{graphicx}
\usepackage{gclc}


\begin{document}

\section{Statement rules}
Statements are relations over constructed objects (for example, two
points coincide, two lines are parallel, two planes are orthogonal).
Each kind of statement introduces some polynomial constraints of the
coordinates of the objects that are involved. No new objects (and
therefore, no new parameters) are introduced. These polynomials are
added to the statement set.

\begin{description}
% ------------------------
\item[$\triangleright$] {\tt equal\_points} $A$ $B$ --- two points $A$
  and $B$ have the same coordinates.

  {\em Polynomials:} The vector equation $\overrightarrow{AB} = 0$
  yields the following three polynomial equations:
  \begin{tabbing}
    $a^x - b^x = 0$\\
    $a^y - b^y = 0$\\
    $a^z - b^z = 0$
  \end{tabbing}

  Note that these polynomials are used only to prove that two points
  are equal, i.e., only as the conclusion polynomials. If it is known
  that two points are equal during construction, then they are given
  the same coordinates, and no polynomials are introduced.

% ------------------------
\item[$\triangleright$] {\tt congruent} $A$ $B$ $C$ $D$ --- Two
  segments, $AB$ and $CD$ are congruent.

  %{\em Input:} Points $A$, $B$, $C$, and $D$.

{\em Polynomials:}
$\overrightarrow{AB} \cdot \overrightarrow{AB} = \overrightarrow{CD} \cdot \overrightarrow{CD}$. \\
This gives the following polynomial equation:

$$({a^x} - {b^x})^2 + ({a^y} - {b^y})^2 + ({a^z} - {b^z})^2 - ({c^x} - {d^x})^2 - ({c^y} - {d^y})^2 - ({c^z} - {d^z})^2 = 0$$

% ------------------------
\item[$\triangleright$] {\tt segments\_in\_ratio} $A$ $B$ $C$ $D$
  $m$ $n$ --- the lengths of segments $AB$ and
  $CD$ are in the given ratio $\frac{m}{n}$ i.e., that
  $\frac{|AB|}{|CD|} = \frac{m}{n}$.

{\em Polynomials:} A single equation obtained from the condition
$n^2 \cdot \overrightarrow{AB} \cdot \overrightarrow{AB} = m^2 \cdot \overrightarrow{CD} \cdot \overrightarrow{CD}$. \\

$\ n^2\cdot(a^x - b^x)^2 + n^2\cdot(a^y - b^y)^2 + n^2\cdot(a^z - b^z)^2 - m^2\cdot(c^x - d^x)^2 - $ \\
\ \ \ $m^2\cdot(c^y - d^y)^2 - m^2\cdot(c^z - d^z)^2 = 0$

{\em Explanation:} The squares of distances between $A$ and $B$ and
between $C$ and $D$ must be in the ratio $\frac{m^2}{n^2}$. Note that
this reduces to congruence when $m = n$.

% ------------------------
\item[$\triangleright$] {\tt midpoint} $M$ $A$ $B$ --- the point
  $M$ is the midpoint of the segment determined by points $A$ and
  $B$.

{\em Polynomials:} The three polynomial equations derived from $\overrightarrow{MA} = \overrightarrow{MB}$:

\begin{tabbing}
$2m^x - a^x - b^x = 0$ \\ 
$2m^y - a^y - b^y = 0$ \\ 
$2m^z - a^z - b^z = 0$
\end{tabbing}

% ------------------------
\item[$\triangleright$] {\tt point\_segment\_ratio} $M$ $A$ $B$ $m$
  $n$ --- checks weather the point $M$ divides segment determined by
  points $A$ and $B$ in ratio determined by $m$ and $n$,
  e.g. $\frac{|MA|}{|MB|} = \frac{m}{n}$.

{\em Polynomials:} The three polynomial equations derived from
$n\cdot \overrightarrow{MA} = m\cdot \overrightarrow{MB}$.

{\em Explanation:} Note that {\tt midpoint $M$ $A$ $B$} can be
expressed as {\tt point\_segment\_ratio} $M$ $A$ $B$ $1$ $1$.

\bigskip

Next we describe some relations that involve both points and lines.

% ------------------------
\item[$\triangleright$] {\tt point\_on\_line} $A$ $P$ $Q$ --- point
  $A$ belongs to the line $P$ $Q$.

{\em Polynomials:} The three polynomials are derived from
$\overrightarrow{PQ} \times \overrightarrow{PA} = 0$.

Note that although there are three polynomial equations, one can
always be derived from the other two. The three polynomials are:

\begin{eqnarray*}
  poly_1 = (a^y - p^y)\cdot (q^z - p^z) - (a^z - p^z)\cdot (q^y - p^y) = 0\\
  poly_2 = (a^z - p^z)\cdot (q^x - p^x) - (a^x - p^x)\cdot (q^z - p^z) = 0\\
  poly_3 = (a^x - p^x)\cdot (q^y - p^y) - (a^y - p^y)\cdot (q^x - p^x) = 0
\end{eqnarray*}


% ------------------------
\item[$\triangleright$] {\tt point\_on\_line} $A$ $l$ --- point $A$ belongs
  to the line $l$.

{\em Polynomials:} If the first approach is used, then this reduces to
$point\_on\_line\ A\ l_A\ l_B$ described above. If the second approach is
used, then the three polynomials are derived from
$\overrightarrow{l_v} \times \overrightarrow{Al_A} = 0$.

In both approaches one polynomial can be derived from the other
two. Second approach polynomials:
\begin{tabbing}
$poly_1 = l^{p_x}\cdot l^{v_y} - a^x\cdot l^{v_y} - l^{p_y}\cdot l^{v_x} + a^y\cdot l^{v_x}$ \\
$poly_2 = l^{p_x}\cdot l^{v_z} - a^x\cdot l^{v_z} - l^{p_z}\cdot l^{v_x} + a^z\cdot l^{v_x}$ \\
$poly_3 = l^{p_y}\cdot l^{v_z} - a^y\cdot l^{v_z} - l^{p_z}\cdot l^{v_y} + a^z\cdot l^{v_y}$
\end{tabbing}

% ------------------------
\item[$\triangleright$] {\tt lines\_intersection} $A$ $P$ $Q$ $M$ $N$
  --- point $A$ is the intersection of lines $PQ$ and $MN$.

{\em Polynomials:} Six polynomials are derived from the conditions
$point\_on\_line\ A\ l_1$ and $point\_on\_line\ A\ l_2$.

Second approach polynomials:


\begin{tabbing}
$a^x - k_1\cdot p^x + k_1\cdot q^x - q^x = 0$ \\
$a^y - k_\cdot p^y + k_1\cdot q^y - q^y = 0$ \\
$a^z - k_1\cdot p^z + k_1\cdot q^z - q^z = 0$ \\
$a^x - k_2\cdot m^x + k_2\cdot n^x - n^x = 0$ \\
$a^y - k_2\cdot m^y + k_2\cdot n^y - n^y = 0$ \\
$a^z - k_2\cdot m^z + k_2\cdot n^z - n^z = 0$
\end{tabbing}

{\em Explanation:} The polynomials were generated using the following
line equations: \\ $x = k_1\cdot (b^x - a^x) + a^x \ \ \ \ y =
k_1\cdot (b^y - a^y) + a^y \ \ \ \ z = k_1\cdot (b^z - a^z) + a^z$ ---
line determined with points $A$ and $B$ \\ $x = k_2\cdot (d^x - c^x) +
c^x \ \ \ \ y = k_2\cdot (d^y - c^y) + c^y \ \ \ \ z = k_2\cdot (d^z -
c^z) + c^z$ --- line determined with points $C$ and $D$.

% ------------------------
\item[$\triangleright$] {\tt lines\_intersection} $A$ $l_1$ $l_2$
  --- point $A$ is the intersection of lines $l_1$ and $l_2$.

  {\em Polynomials:} If the second approach is used, six polynomials
  are derived from the conditions $point\_on\_line\ A\ l_1$ and
  $point\_on\_line\ A\ l_2$.
  
  Second approach polynomials:
  \begin{tabbing}
    $a^x - k_1\cdot l_1^{v_x} - l_1^{p_x} = 0$ \\
    $a^y - k_1\cdot l_1^{v_y} - l_1^{p_y} = 0$ \\
    $a^z - k_1\cdot l_1^{v_z} - l_1^{p_z} = 0$ \\
    $a^x - k_2\cdot l_2^{v_x} - l_2^{p_x} = 0$ \\
    $a^y - k_2\cdot l_2^{v_y} - l_2^{p_y} = 0$ \\
    $a^z - k_2\cdot l_2^{v_z} - l_2^{p_z} = 0$
  \end{tabbing}

  {\em Explanation:} 
  Since the point $a$ should belong to both lines $l_1$ (given by the
  point $\overrightarrow{l_1^p}$ and the vector
  $\overrightarrow{l_1^v}$) and $l_2$ (given by the point
  $\overrightarrow{l_2^p}$ and the vector $\overrightarrow{l_2^v}$) it
  must satisfy their parametric equations, i.e., it must hold that
  $\overrightarrow{A} = \overrightarrow{l_1^p} + k_1 \cdot
  \overrightarrow{l_1^v}$
  and
  $\overrightarrow{A} = \overrightarrow{l_2^p} + k_2 \cdot
  \overrightarrow{l_2^v}$.


% ------------------------
\item[$\triangleright$] {\tt orthogonal\_lines} $A$ $B$ $C$ $D$ ---
  the line $AB$ is orthogonal on the line $CD$.

  {\em Polynomials:} A necessary condition is
  $\overrightarrow{AB} \cdot \overrightarrow{CD} = 0$ and it yields
  one polynomial. However, this conditions is not enough since it is
  only checked weather the appropriate vectors of the lines are
  orthogonal but lines itself could be skew. So, one more condition
  must be added in order to ensure that lines really intersect, and it
  can be that vectors $\overrightarrow{AB}$, $\overrightarrow{AC}$,
  and $\overrightarrow{CD}$ are coplanar, that is easily expressed
  using scalar triple product:
  $\overrightarrow{AC} \times (\overrightarrow{AB} \cdot
  \overrightarrow{CD}) = 0$, and this yields three more polynomials.

% ------------------------
\item[$\triangleright$] {\tt orthogonal\_lines} $l$ $m$ --- the lines
  $l$ and $m$ are orthogonal.

  {\em Polynomials:} If the second approach is used, the direction
  vectors of the lines are given as the input, and the polynomials are
  derived from $\overrightarrow{l_v} \cdot \overrightarrow{m_v} = 0$
  and
  $\overrightarrow{m_Am_B} \times (\overrightarrow{l_v} \cdot
  \overrightarrow{m_v}) = 0$.


% ------------------------
\item[$\triangleright$] {\tt parallel\_lines} $A$ $B$ $C$ $D$ ---
  lines $AB$ and $CD$ are parallel.

  {\em Polynomials:} The three polynomials are derived from
  $\overrightarrow{AB} \times \overrightarrow{CD} = 0$.

% ------------------------
\item[$\triangleright$] {\tt parallel\_lines} $p$ $q$ --- lines $p$
  and $q$ are parallel.

  {\em Polynomials:} If the second approach is used, the three
  polynomials derived from
  $\overrightarrow{p_v} \times \overrightarrow{q_v} = 0$.

Second approach polynomials:
\begin{tabbing}
$p^{v_y}\cdot q^{v_z} - p^{v_z}\cdot q^{v_y} = 0$ \\
$p^{v_z}\cdot q^{v_x} - p^{v_x}\cdot q^{v_z} = 0$ \\
$p^{v_x}\cdot q^{v_y} - p^{v_y}\cdot q^{v_x} = 0$
\end{tabbing}

% ------------------------
\item[$\triangleright$] {\tt equal\_angles} $A$ $O_1$ $B$ $C$ $O_2$ $D$ --- angles $\angle AO_1B$ and $\angle
CO_2D$ are equal.

{\em Polynomials:} Polynomials for this condition could be derived
using trigonometry condition:
$$\cos{\angle AO_1B} = \cos{\angle CO_2D}$$

and cosine of an angle can be determined using following equation:
$$\cos{\angle AO_1B} = \frac{\overrightarrow{AO_1}\cdot
  \overrightarrow{BO_1}}{2|AO_1||BO_1|}.$$

Polynomial equation for $\cos{\angle CO_2D}$ is similar. However,
distances ($|AO_1|$ and $|BO_1|$) are calculated square roots and that
cannot be processed by algebraic theorem provers (that accept only
polynomial equations). Thus, squares of the cosines of an angles must
be compared --- $\cos^2{\angle AO_1B} = \cos^2{\angle CO_2D}$.
Unfortunately, this makes the condition weakened since squares of
cosines of accute and obtuse angle could be same although angles
itself are not equal. The final polynomial is derived from
$$(\overrightarrow{AO_1}\cdot  \overrightarrow{BO_1})^2|CO_2|^2|DO_2|^2 = (\overrightarrow{CO_2}\cdot  \overrightarrow{DO_2})^2|AO_1|^2|BO_1|^2.$$

The polynomial derived from this equation is quite complex and
challenging for algebraic theorem provers and it can be dissembled
onto more simpler ones:
\begin{tabbing}
$x_1 - (a^x - o_1^x)\cdot(b^x - o_1^x) - (a^y - o_1^y)\cdot(b^y - o_1^y) - (a^z - o_1^z)\cdot(b^z - o_1^z) = 0$ \\
$x_2 - (c^x - o_2^x)\cdot(d^x - o_2^x) - (c^y - o_2^y)\cdot(d^y - o_2^y) - (c^z - o_2^z)\cdot(d^z - o_2^z) = 0$ \\
$x_3 - (a^x - o_1^x)^2 - (a^y - o_1^y)^2 - (a^z - o_1^z)^2 = 0$ \\
$x_4 - (b^x - o_1^x)^2 - (b^y - o_1^y)^2 - (b^z - o_1^z)^2 = 0$ \\
$x_5 - (c^x - o_2^x)^2 - (c^y - o_2^y)^2 - (c^z - o_2^z)^2 = 0$ \\
$x_6 - (d^x - o_2^x)^2 - (d^y - o_2^y)^2 - (d^z - o_2^z)^2 = 0$ \\
$x_1^2\cdot x_5 \cdot x_6 - x_2^2\cdot x_3\cdot x_4 = 0$
\end{tabbing}

\bigskip

Next we describe some relations that also involve planes.


% ------------------------
\item[$\triangleright$] {\tt point\_in\_plane} $A$ $P$ $Q$ $R$
  --- the point $A$ belongs to the plane determined by points $P$,
  $Q$, and $R$.

  {\em Polynomials:} The polynomial
$$\overrightarrow{PA}\cdot (\overrightarrow{PQ} \times \overrightarrow{PR}) = 0.$$

% ------------------------
\item[$\triangleright$] {\tt point\_in\_plane} $A$ $\pi$ --- the
  point $A$ belongs to the plane $\pi$.

  {\em Polynomials:} If the second approach is used, the polynomial
  $\overrightarrow{\pi_v} \cdot \overrightarrow{A} + \pi^{d} = 0$.

Second approach polynomials:
\begin{tabbing}
$\pi^{v_x}\cdot a^x + \pi^{v_y}\cdot a^y + \pi^{v_z}\cdot a^z + \pi^{d} = 0$
\end{tabbing}


% ------------------------
\item[$\triangleright$] {\tt parallel\_planes} $A$ $B$ $C$ $P$ $Q$ $R$
  --- the plane determined by points $A$, $B$, and $C$, and the plane
  determined by points $P$, $Q$, and $R$ are parallel.

{\em Polynomials:} If the first approach is used, polynomials are
derived from:

$$\overrightarrow{PQ}\cdot \overrightarrow{AC} \times \overrightarrow{BA} = 0$$
$$\overrightarrow{PR}\cdot \overrightarrow{AC} \times \overrightarrow{AB} = 0$$

Denote the plane determined by $A$, $B$, and $C$ by $\alpha$, and the
plane determined by $P$, $Q$, and $R$ by $\beta$. These two equations
are derived from the conditions that vectors of all lines parallel to
$\beta$ plane are orthogonal to the normal vector of the plane
$\beta$. Since $\beta$ and $\alpha$ should be parallel, vectors of
lines in $\beta$ should be orthogonal to the normal vector of
$\alpha$, e.g. to the
$\overrightarrow{AC} \times \overrightarrow{BA}$.  However,
polynomials gained from these equations are quite complex (and can
negatively effect algebraic theorem provers). For example, the first
polynomial is:
$$(P^x - Q^x)(A^y - B^y)(A^z - C^z) + (A^x - B^x)(A^y - C^y)(P^z - Q^z) \ +$$
$$(P^x - Q^y)(A^z - B^z)(A^x - C^x) - (P^z - Q^z)(A^y - B^y)(A^x - C^x) \ -$$
$$(A^z - B^z)(A^y - C^y)(P^x - Q^x) - (P^x - Q^y)(A^x - B^x)(A^z - C^z) = 0$$


% ------------------------
\item[$\triangleright$] {\tt parallel\_planes} $\alpha$ $\beta$ ---
  planes $\alpha$ and $\beta$ are parallel.

{\em Polynomials:} 
If the second approach is used, the three polynomials are derived from
$\overrightarrow{\alpha_v} \times \overrightarrow{\beta_v} = 0.$ For
the comparison, one of the three polynomials derived using this
approach is:
$$\alpha_{v^y}\beta_{v^z} - \alpha_{v^z}\beta_{v^y} = 0,$$
and it is significantly simpler than the one in the first approach.

\item[$\triangleright$] {\tt orthogonal\_planes} $A$ $B$ $C$ $P$ $Q$ $R$
  --- the plane determined by points $A$, $B$, and $C$, and the plane
  determined by points $P$, $Q$, and $R$ are orthogonal.

  {\em Polynomials:} The polynomial is derived from
  $(\overrightarrow{AB} \times \overrightarrow{AC}) \cdot
  (\overrightarrow{PQ} \times \overrightarrow{PR}) = 0$.

Second approach polynomials:
\begin{tabbing}
$\alpha^{v_y}\cdot \beta^{v_z} - \alpha^{v_z}\cdot \beta^{v_y} = 0$ \\
$\alpha^{v_z}\cdot \beta^{v_x} - \alpha^{v_x}\cdot \beta^{v_z} = 0$ \\
$\alpha^{v_x}\cdot \beta^{v_y} - \alpha^{v_y}\cdot \beta^{v_x} = 0$
\end{tabbing}


% ------------------------
\item[$\triangleright$] {\tt orthogonal\_planes} $\alpha$ $\beta$ ---
  checks weather two planes, $\alpha$ and $\beta$ are orthogonal.


If the second approach is used, the polynomial is
$\overrightarrow{\alpha_v} \cdot \overrightarrow{\beta_v} = 0$.

Second approach polynomial:
\begin{tabbing}
$\alpha^{v_x}\cdot \beta^{v_x} + \alpha^{v_y}\cdot \beta^{v_y} + \alpha^{v_z}\cdot \beta^{v_z} = 0$
\end{tabbing}

% ------------------------
\item[$\triangleright$] {\tt parallel\_line\_plane} $A$ $B$ $P$ $Q$ $R$ ---
  line $AB$ and plane $PQR$ are parallel.

  {\em Polynomials:} The polynomial is determined from
  $\overrightarrow{AB} \cdot (\overrightarrow{PQ \times PR}) = 0$.

{\em Note:} This statement also holds if line belongs to the plane.

% ------------------------
\item[$\triangleright$] {\tt parallel\_line\_plane} $p$ $\alpha$ ---
  line $p$ and plane $\alpha$ are parallel.

  {\em Polynomials:} If the second approach is used, then polynomial
  $\overrightarrow{p_v} \cdot \overrightarrow{\alpha_v} = 0$.

{\em Note:} This statement also holds if line belongs to the plane.

Second approach polynomial:
\begin{tabbing}
$p^{v_x}\cdot \alpha^{v_x} + p^{v_y}\cdot \alpha^{v_y} + p^{v_z}\cdot \alpha^{v_z} = 0$
\end{tabbing}


% ------------------------
\item[$\triangleright$] {\tt orthogonal\_line\_plane} $A$ $B$ $P$ $Q$
  $R$ --- the line $AB$ and plane $PQR$ are orthogonal.

  {\em Polynomials:} Two polynomials are derived from
  $\overrightarrow{AB} \cdot \overrightarrow{PQ} = 0$ and
  $\overrightarrow{AB} \cdot \overrightarrow{PR} = 0$.

% ------------------------
\item[$\triangleright$] {\tt orthogonal\_line\_plane} $p$ $\alpha$ ---
  the line $p$ and plane $\alpha$ are orthogonal.

  {\em Polynomials:} If the second approach is used the three
  polynomials derived from
  $\overrightarrow{p_v} \times \overrightarrow{\alpha_v} = 0$.

  Second approach polynomials:
\begin{tabbing}
$\alpha^{v_y}\cdot p^{v_z} - \alpha^{v_z}\cdot p^{v_y} = 0$\\
$\alpha^{v_z}\cdot p^{v_x} - \alpha^{v_x}\cdot p^{v_z} = 0$\\
$\alpha^{v_x}\cdot p^{v_y} - \alpha^{v_y}\cdot p^{v_x} = 0$
\end{tabbing}


% ------------------------
\item[$\triangleright$] {\tt line\_plane\_intersection\_} $A$ $M$ $N$
  $P$ $Q$ $R$ --- point $A$ is the intersection of the line $MN$ and
  plane $PQR$.

  {\em Polynomials:} Polynomials are derived from {\tt point\_on\_line
    $A$ $M$ $N$} and {\tt point\_in\_plane $A$ $P$ $Q$ $R$}.

% ------------------------
\item[$\triangleright$] {\tt line\_plane\_intersection} $A$ $l$ $\pi$ ---
  point $A$ is the intersection of line $l$ and plane $\pi$.

  {\em Polynomials}: Polynomials are derived from {\tt point\_on\_line
    $A$ $l$} and {\tt point\_in\_plane $A$ $\pi$}.

Second approach polynomials:
\begin{tabbing}
$a^x - k\cdot l^{v_x} - l^{p_x} = 0$ \\
$a^y - k\cdot l^{v_y} - l^{p_y} = 0$ \\
$a^z - k\cdot l^{v_z} - l^{p_z} = 0$ \\
$\pi^{v_x}\cdot a^x + \pi^{v_y}\cdot a^y + \pi^{v_z}\cdot a^z + \pi^{d} = 0$
\end{tabbing}
{\em Explanation:}  In the generated polnomials is used line
ratio denoted with $k$. For determing polinomials is used line equation:
$$x = k\cdot l^{v_x} + l^{p_x}\ \ \ \ \ y = k\cdot l^{v_y} +
l^{p_y}\ \ \ \ \ z = k\cdot l^{v_z} + l^{p_z}$$ and the fact that
point $A$ belongs to the line $l$. It is also used the fact that the
point belongs to the plane $\pi$ and it is used plane equation: 
$$\pi^{v_x}\cdot x + \pi^{v_y}\cdot y + \pi^{v_z}\cdot z + \pi^{d} =
0$$ There are four polynomials since there are four unknown variables,
tree variables are coordinates of the point, and one variable is line
ratio.

% ------------------------
\item[$\triangleright$] {\tt line\_in\_plane} $AB$ $PQR$ --- the line
  $AB$ is in the plane $PQR$.

  {\em Polynomials:} Polynomials are derived from two conditions {\tt
    point\_in\_plane} $A$ $P$ $Q$ $R$ and {\tt point\_in\_plane} $B$
  $P$ $Q$ $R$.

% ------------------------
\item[$\triangleright$] {\tt line\_in\_plane} $l$ $\alpha$ --- the $l$
  is in the plane $\alpha$.

  {\em Polynomials:} If the second approach is used, polynomials are
  derived from {\tt point\_in\_plane} $l_A$ $\pi$ and
  $l_v \cdot \alpha_v = 0$.

Second approach polynomials:
\begin{tabbing}
$l^{v_x}\cdot \alpha^{v_x} + l^{v_y}\cdot \alpha^{v_y} + l^{v_z}\cdot \alpha^{v_z} = 0$ \\
$\alpha^{v_x}\cdot l^{p_x} + \alpha^{v_y}\cdot l^{p_y} + \alpha^{v_z}\cdot l^{p_z} + \alpha^{d} = 0$
\end{tabbing}

\bigskip
Some theorems can involve spheres. 

% ------------------------
\item[$\triangleright$] {\tt point\_on\_sphere} $A$ $\mathcal{S}$
  --- point $A$ belongs to the sphere $\mathcal{S}$.

{\em Polynomials:} Sphere $\mathcal{S}$ is defined with its center $O$
and radius $r$. Thus, polynomial is derived from
$\overrightarrow{AO}\cdot\overrightarrow{AO} = r^2$.

% ------------------------
\item[$\triangleright$] {\tt equal\_multiple\_four\_numbers} $m$ $n$
  $p$ $q$ -- checks weather it is true $m*n = p*q$ where $m$, $n$, $p$
  and $q$ are four integer numbers.

\end{description}

\section{Constructions of points}

\begin{description}

% ------------------------
\item[$\triangleright$] $A$ = {\tt make\_point} --- a free point.

  {\em Introduced objects and parameters}: a point $A$ with the three
  fresh parameters $A_x$, $A_y$, and $A_z$ representing its
  coordinates.

  {\em Polynomials:} No polynomials are introduced.

% ------------------------
\item[$\triangleright$] $A$ $=$ {\tt make\_point\_on\_line} $l$ ---
  construct a semi-free point $A$ that belongs to the line $l$.
  
  {\em Introduced objects and parameters}: introduces a point $A$ with
  three fresh parameters $a^x$, $a^y$ and $a^z$ representing its
  coordinates.

  {\em Polynomials:} derived from {\tt point\_on\_line $A$ $l$}.

% ------------------------
\item[$\triangleright$] $A$ = {\tt make\_point\_in\_plane} $\pi$ ---
  construct a semi-free point $A$ that belongs to the plane $\pi$.

  {\em Introduced objects and parameters}: introduces a point $A$ with
  three fresh parameters $a^x$, $a^y$ and $a^z$ representing its
  coordinates.
  
  {\em Polynomials:} derived from {\tt point\_in\_plane $A$ $\pi$}.

% ------------------------
\item[$\triangleright$] $A$ = {\tt make\_lines\_intersection} $l_1$
  $l_2$ --- point $A$ is the intersection of lines $l_1$ and $l_2$.

  {\em Introduced objects and parameters}: a point $A$ with three
  fresh parameters $a^x$, $a^y$ and $a^z$ representing its
  coordinates.

  {\em Polynomials:} derived from {\tt lines\_intersection $A$ $l_1$ $l_2$}.

% ------------------------
\item[$\triangleright$] $A$ = {\tt make\_line\_plane\_intersection}
  $l$ $\pi$ --- point $A$ is the intersection of line $l$ and plane
  $\pi$.

  {\em Introduced objects and parameters}: a point $A$ with three
  fresh parameters $a^x$, $a^y$ and $a^z$ representing its
  coordinates.

  {\em Polynomials:} derived from {\tt line\_plane\_intersection $A$ $l$ $\pi$}.

\bigskip

We also support isometric transformations. For example, in most tasks,
translation in the z-axis direction is used.

% ------------------------
\item[$\triangleright$]  $A$ = {\tt translate\_z} $O$ $d$ --- construct a
  point by translating point $A$ by some parameter along $z$-axis.

  {\em Introduced objects and parameters:} point $A$ with the
  coordinates $(o^x, o^y, a^z)$, for a single fresh parameter $a^z$.

  {\em Polynomial:} A polynomial derived from
  $\overrightarrow{AO} = (0, 0, d)$.

\bigskip
 
Support for integer numbers.

% ------------------------
\item[$\triangleright$] {\tt make\_distance\_number} $n$ $A$ $B$ ---
  constructs the variable with name $n$ representing square distance
  between points $A$ and $B$.

  $n = \overrightarrow{AB} \cdot \overrightarrow{AB}$
\end{description}


\paragraph{Constructions of lines}
Although the construction of a free line could be easily introduced,
for simplicity, we have chosen to support only free points (and a free
line can be constructed as a line trough two free points).

\begin{description}
% ------------------------
\item[$\triangleright$] $l$ $=$ {\tt make\_line\_trough\_points} $A$
  $B$ --- for two given points $A$ and $B$ constructs the line $l$
  that contains them.

  {\em Introduced objects and parameters:} if the first approach is
  used, no new parameters are introduced and the name $l$ is
  associated with the pair of points $A$ and $B$. If the second
  approach is used, fresh parameters $l^{v_x}$, $l^{v_y}$, $l^{v_z}$
  that represent the direction vector of the are introduced. There is
  no need to introduce the parameters for the point $l^A$, as it can
  be associated either to the point $A$ or the point $B$.

  {\em Polynomials:} If the first approach is used no polynomials are
  needed. If the second approach is used, the polynomials are derived
  from {\tt point\_on\_line $B$ $l$} (if $A$ is chosen for $l^A$).

Second approach polynomials:
\begin{tabbing}
$l^{v_x} - a^x + b^x = 0$ \\
$l^{v_y} - a^y + b^y = 0$ \\
$l^{v_z} - a^z + b^z = 0$
\end{tabbing}

% ------------------------
\item[$\triangleright$] $l$ = {\tt make\_line\_orthogonal\_on\_plane}
  $\pi$ $A$ -- for a given plane $\pi$ and a point $A$ constructs the
  line $l$ that is orthogonal on the given plane $\pi$ and contains
  the point $A$.

  {\em Introduced objects and parameters:} if the first approach is
  used then first point $l_A$ determining the line is point $A$. The
  fresh symbolic coordinates of the second point $l_B$ are introduced.

  If the second approach is used line $l$ is determined by the
  parameters $(\pi^{v_x}, \pi^{v_y}, \pi^{v_z}, a^x, a^y, a^z)$ (its
  direction vector is the normal vector of the plane, and its point is
  the point $A$).

  {\em Polynomials:} If the first approach is used, polynomials are
  determined by
  $l_B = l_A + \overrightarrow{\pi_A\pi_B} \times
  \overrightarrow{\pi_A\pi_C}$.

  If the second approach is used, no polynomials are generated.

% ------------------------
\item[$\blacktriangleright$] $l$ = {\tt make\_line\_orthogonal\_on\_plane\_} $A$ $B$ $C$ $D$\\

{\em Description:} For given points $A$, $B$, $C$ and $D$ determines
line $l$ witch is orhogonal on the plane determined with points $A$,
$B$ and $C$ and incides with the point $A$.

Second approach polynomials:
\begin{tabbing}
$l^{v_x} - b^y\cdot c^z + b^y\cdot a^z + a^y\cdot c^z + b^z\cdot c^y - b^z\cdot a^y - a^z\cdot c^y = 0$ \\
$l^{v_y} - b^x\cdot c^z - b^x\cdot a^z - a^x\cdot c^z - b^z\cdot c^x + b^z\cdot a^x + a^z\cdot c^x = 0$ \\
$l^{v_z} - b^x\cdot c^y + b^x\cdot a^y + a^x\cdot c^y + b^y\cdot c^x - b^y\cdot a^x - a^y\cdot c^x = 0$ 
\end{tabbing}

{\em Explanation:} The parameters for the point on the line can be the
point $A$. Line vector is the same as the vector of the plane $ABC$,
and can be determined using formula:
$$(l^{v_x}, l^{v_y}, l^{v_z}) = \begin{vmatrix} i & j & k \\ b^x - a^x
  & b^y - a^y & b^z - a^z \\ c^x - a^x & c^y - a^y & c^z - a^z
\end{vmatrix}$$
\end{description}


\section{Constructions of the plane}

\begin{description}
% ------------------------
\item[$\triangleright$] $\pi$ = {\tt make\_plane\_trough\_points} $A$
  $B$ $C$ --- for three given points $A$, $B$ and $C$ constructs the
  plane $\pi$ containing all of them.

  {\em Introduced objects and parameters:} if the first approach is
  used, no new variables are introduced and the name $\pi$ is
  associated with the triple of points $A$, $B$, and $C$. If the
  second approach is used, then plane $\pi$ is determined with fresh
  symbolic parameters $(\pi^{v_x}, \pi^{v_y}, \pi^{v_z}, \pi^{d})$.

  {\em Polynomials:} If the second approach is used, polynomials are
  derived from
  $\overrightarrow{\pi} = \overrightarrow{AB} \times
  \overrightarrow{AC}$ and {\tt point\_on\_plane $A$ $\pi$}.

Second approach polynomials:
\begin{tabbing}
$\pi^{v_x} - b^y\cdot c^z + b^y\cdot a^z + a^y\cdot c^z + b^z\cdot c^y - b^z\cdot a^y - a^z\cdot c^y = 0$ \\
$\pi^{v_y} - b^x\cdot c^z - b^x\cdot a^z - a^x\cdot c^z - b^z\cdot c^x + b^z\cdot a^x + a^z\cdot c^x = 0$ \\
$\pi^{v_z} - b^x\cdot c^y + b^x\cdot a^y + a^x\cdot c^y + b^y\cdot c^x - b^y\cdot a^x - a^y\cdot c^x = 0$ \\
$\pi^{d} + \pi^{v_x}\cdot a^x + \pi^{v_y}\cdot a^y + \pi^{v_z}\cdot a^z$
\end{tabbing}

{\em Explanation:} The normal vector of the line can be obtained as a
vector product of the vectors $\overrightarrow{AB}$ and
$\overrightarrow{AC}$. Since $(b^x - a^x, b^y - a^y, b^z - a^z)$ and
$(c^x - a^x, c^y - a^y, c^z - a^z)$ are vectors of two lines
determined with points $A$ and $B$, and points $C$ and $D$, the first
three polynomials are derived using the formula:
$$(\pi^{v_x}, \pi^{v_y}, \pi^{v_z}) = \begin{vmatrix} i & j & k \\ b^x
  - a^x & b^y - a^y & b^z - a^z \\ c^x - a^x & c^y - a^y & c^z - a^z
\end{vmatrix}$$

The last parameter of the plane $\pi^{d}$ is obtained from the
condition that the point $A$ is in the plane $\pi$, and satisfies its
equation. Note that this equation uses the symbolic parameters
$\pi^{v_x}$, $\pi^{v_y}$ and $\pi^{v_z}$ that are introduced in this
construction step.


% ------------------------
\item[$\triangleright$] $\pi$ = {\tt
    make\_plane\_orthogonal\_on\_plane\_containing\_line} $\alpha$ $l$
  --- for given plane $\alpha$ and line $l$ constructs plane with line
  $l$ and orthogonal to the plane $\alpha$.

  {\em Introduced objects and parameters:} if the first approach is
  used, no new variables are introduced and the name $\pi$ is
  associated with the triple of points $l_A$, $l_B$, and new fresh
  point $\pi_C$. If the second approach is used, then plane $\pi$ is
  determined with fresh symbolic parameters
  $(\pi^{v_x}, \pi^{v_y}, \pi^{v_z}, \pi^{d})$.

  {\em Polynomials:} If the first approach is used polynomials are
  derived from $(\overrightarrow{\pi_Cl_A} \times
  \overrightarrow{\pi_Cl_B}) \cdot (\overrightarrow{\alpha_A\alpha_B}
  \times \overrightarrow{\alpha_A\alpha_C}) = 0$. If the second
  approach is used, polynomials are derived from $\overrightarrow{\pi}
  = \overrightarrow{\alpha} \times \overrightarrow{l}$ and {\tt
    point\_on\_plane $l_A$ $\pi$}.

Second approach polynomials:
\begin{tabbing}
$\pi^{v_x} - \alpha^{v_y}\cdot l_z + \alpha^{v_z}\cdot l_y = 0$ \\
$\pi^{v_y} - \alpha^{v_z}\cdot l_x + \alpha^{v_x}\cdot l_z = 0$ \\
$\pi^{v_z} - \alpha^{v_x}\cdot l_y + \alpha^{v_y}\cdot l_x = 0$ \\
$\pi^{v_x}\cdot l^{p_x} + \pi^{v_y}\cdot l^{p_y} + \pi^{v_z}\cdot l^{p_z} + \pi^{d} = 0$
\end{tabbing}
{\em Explanation:} These polynomials were derived using formula:
$$ (\pi^{v_x}, \pi^{v_y}, \pi^{v_z}) = \begin{vmatrix}
i & j & k \\
\alpha^{v_x} & \alpha^{v_y} & \alpha^{v_z} \\
l^{v_x} & l^{v_y} & l^{v_z}
\end{vmatrix}
$$ and the condition that point $(l^{p_x}, l^{p_y}, l^{p_z})$ also
lies in plane $\pi$.
\end{description}


\section{Solid construction}

\begin{description}
% ------------------------
\item[$\triangleright$] $A$ $B$ $C$ $D$ $A_1$ $B_1$ $C_1$ $D_1$ = {\tt
    make\_cube} --- construct the cube in the canonical position, with
  the edge length equal to $1$.

  {\em Introduced objects and parameters:} Points $A(0, 0, 0)$,
  $B(1, 0, 0)$, $C(1, 1, 0)$, $D(0, 1, 0)$, $A_1(0, 0, 1)$,
  $B_1(1, 0, 1)$, $C_1(1, 1, 1)$ and $D(0, 1, 1)$. Since the cube is
  in the canonical position, no symbolic variables are introduced.

  {\em Explanation:} No polynomials are generated.

% ------------------------
\item[$\triangleright$] $A$ $B$ $C$ $D$ = {\tt
    make\_tetrahedron} \label{tetrahedron} --- construct the
  tetrahedron in the
  canonical position.

  {\em Introduced objects and parameters:} The vertices of the
  tetrahedron have the coordinates $A(0, 0, 0)$, $B(1, 0, 0)$,
  $C(c^x, c^y, 0)$ and $D(c^x, d^y, d^z)$, with the four fresh
  parameters $c^x$, $c^y$, $d^y$, and $d^z$.

\begin{tabbing}
{\em Polynomials:} \= $poly_1 = 2\cdot c^x - 1$ \\
                   \> $poly_2 = 2\cdot {c^y}^2 - 3$ \\
                   \> $poly_3 = 3\cdot d^y - c^y$ \\
                   \> $poly_4 = 3\cdot {d^z}^2 - 2$
\end{tabbing}

{\em Explanation:} $c^x = \frac{1}{2}$, $c^y = \frac{\sqrt{3}}{2}$,
$d^y = \frac{\sqrt{3}}{6} = \frac{c^y}{3}$,
$d^z = \frac{\sqrt{2}}{\sqrt{3}}$. Note that all our objects always
have either symbolic or integer parameters and polynomials must always
have integer coefficients, so irrational values (and even fractions)
must be introduced using polynomials.

% ------------------------
\item[$\triangleright$] $A$ $B$ $C$ $D$ $S$ = {\tt make\_pyramid} ---
   construct a regular four--side pyramid in the canonical position --
   the base is a unit square in the $xOy$ plane, its lateral edges are
   equal in length, and the height is not constrained.
   {\em Introduced objects and parameters:} Points $A(0, 0, 0)$,
   $B(1, 0, 0)$, $C(1, 1, 0)$, $D(0, 1, 0)$ and $S(s^x, s^y, s^z)$, for
   three fresh parameters $s^x$, $s^y$ and $s^z$.

 \begin{tabbing}
 {\em Polynomials:} \= $poly_1 = 2\cdot s^x - 1$ \\
                    \> $poly_2 = 2\cdot s^y - 1$
 \end{tabbing}

 {\em Explanation:} The projection of the apes is $(s^x, s^y, 0)$ and
 it lies in the middle of the square, so $s^x = s^y =
 \frac{1}{2}$. Note that $s^z$ is not constrained.

% ------------------------
\item[$\triangleright$] $\mathcal{S}$ = {\tt make\_sphere} $A$ $B$ $C$
  $D$ --- construct a sphere $\mathcal{S}$ trough four given points
  $A$, $B$, $C$, $D$.
  
 {\em Introduced objects and parameters:} the sphere is determined by
 its center $O = (O^x, O^y, O^z)$ and its radius $r$ --- four fresh
 parameters $O^x$, $O^y$, $O^z$ and $r$ are introduced.

{\em Polynomials:} Polynomials are derived from
$\overrightarrow{AO}\cdot\overrightarrow{AO} = r^2$,
$\overrightarrow{BO}\cdot\overrightarrow{BO} = r^2$,
$\overrightarrow{CO}\cdot\overrightarrow{CO} = r^2$, and
$\overrightarrow{DO}\cdot\overrightarrow{DO} = r^2$.

% ------------------------
 \item[$\triangleright$] $A$ $B$ $C$ $D$ = {\tt make\_square} ---
   construct a square in the canonical position: the base is a unit
   square in the $xOy$ plane, its lateral edges are equal in length,
   and the height is not constrained.

   {\em Introduced objects and parameters:} Points $A(0, 0, 0)$,
   $B(1, 0, 0)$, $C(1, 1, 0)$, $D(0, 1, 0)$.

% ------------------------
 \item[$\triangleright$] $A$ $B$ $C$ = {\tt
     make\_equilateral\_triangle} --- construct a equilateral triangle
   in the canonical position; it is in $xOy$ plane, one point is in
   origin, and another point is on
   $x$-axis. \\
   {\em Introduced objects and parameters:} Points $A(0, 0, 0)$,
   $B(1, 0, 0)$, $C(c^x, c^y, 0)$, for two fresh parameters $c^x$ and
   $c^y$.

 \begin{tabbing}
 {\em Polynomials:} \= $poly_1 = 2\cdot c^x - 1$ \\
                    \> $poly_2 = 4\cdot c^y - 3$
 \end{tabbing}

% ------------------------
\item[$\triangleright$] $A_1$ $A_2$ $A_3$ $A_4$ $A_5$ $A_6$ = {\tt
    make\_regular\_hexagon} --- construct a regular hexagon in the
  canonical position; it is in $xOy$ plane, one point is in origin,
  and another
  point is on $x$-axis. \\
  {\em Introduced objects and parameters:} Points $A_1(0, 0, 0)$,
  $A_2(1, 0, 0)$, $A_3(a_3^x, a_3^y, 0)$, $A_4(1, a_4^y, 0)$,
  $A_5(0, a_4^y, 0)$ and $A_6(a_6^x, a_3^y, 0)$, for four fresh
  parameters $a_3^x$, $a_3^y$, $a_4^y$ and $a_6^x$.

\begin{tabbing}
{\em Polynomials:} \= $poly_1 = 2\cdot a_3^x - 3$ \\
                   \> $poly_2 = 4(a_3^y)^2 - 3$ \\
                   \> $poly_3 = a_4^y - 3$ \\
                   \> $poly_4 = 2a_6^x - 1$
\end{tabbing}
\end{description}


\end{document}

